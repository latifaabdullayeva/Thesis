\chapter{Related Work}
\label{ch:related-work}
This chapter presents the background publications related and applied in our research.
Sections~\ref{sec:Ubiquitous Computing}, ~\ref{sec:Internet of Things}, and ~\ref{sec:Social Internet of Things}
provide an introduction of basic concepts related to the system.
Section~\ref{sec:Autonomous interaction of things} presents a system that inspired us to expand
and use in a user study.
Starting from section~\ref{sec:The Theory of Proxemics}, we discuss some methodologies
that help us to expand the existing system.
Each section presents the related publications, gives an example of projects where these concepts
were used and describe how we used these concepts in our study.

\section{Ubiquitous Computing}
\label{sec:Ubiquitous Computing}
The attempts to make technologies invisible in the background of people’s life led to the emergence
of a new approach in the area of Information Technology whereby making the term
Ubiquitous Computing prominent in recent years.
The “Ubiquitous Computing” was initially put forward by Mark Weiser
in “The computer for the 21st Century” ~\cite{weiser1999computer}.
In this paper, the author touched two issues related to the concept of
Ubiquitous Computing such as location and scale.

The traditional computers which existed before the introduction of this
paradigm had no idea about their location.
The location-aware system may have information about how far or close it is from other
objects and may even later be able to adapt its behavior accordingly.
An example application that leveraged the location-aware paradigm was introduced by
Hupfeld and Berge in their RAUM system ~\cite{hupfeld2000spatially}.
The authors claim that information about the location of objects plays
a more important role than their identities.
They explain the essence of location by giving an example of people who prefer to communicate
while standing in front of the person who participates in the conversation, rather than turning their backs on him.
With the help of the concept of Ubiquitous Computing, the system presented in our
research uses the location information to select a communication partner.

Another issue related to the concept of Ubiquitous Computing is the scale,
that is, systems of various sizes serve different purposes.
In the context of our system, mascot, tablet, lamp, and speakers are all in
different sizes and, therefore, perform different tasks.
Moreover, the size of objects is also reflected in its location, for example, a lamp, compared
to other devices has a larger size, which limits its location to one point, whereas the mascot
which is a pocket-size phone allows changing the location depending on the location of its owner.

\section{Internet of Things}
\label{sec:Internet of Things}
The rapid development of electronics led to the emergence of the concept of “Internet of Things”.
IoT can be both ubiquitous and non-ubiquitous technologies.
Moreover, in the context of Ubiquitous
Computing, IoT adds a new dimension to the interaction between objects: from any time, any place
connectivity for everyone, we will have connectivity for anything ~\cite{tan2010future}.
Thus, in comparison to Ubiquitous technologies, IoT focuses not only on the interaction
between humans and devices but also on the devices themselves.

The idea of IoT was first proposed by Kevin Ashton in 1999 ~\cite{ashton2009internet}
by linking the idea of RFID (Radio Frequency Identification) to the topic of the Internet.
We can characterize IoT as one big network where all devices can share information about their status with
each other allowing us to achieve deeper automation and integration within a system.
In the “Internet of Things: A Literature Review” paper~\cite{madakam2015internet}, authors describe
the genesis of the term “IoT” which help us to understand the general concept behind it and
corresponding key technologies that it uses.
They explained the concept by dividing the definition of IoT into two components: "Internet"
as a global system of interconnected computer networks that use the Internet protocol to serve users worldwide
and "Thing" as real objects in the physical or material world.
This explanation helps us to understand that inanimate objects such as lamps, speakers,
etc can communicate with other objects with the help of the Internet without any explicit human instructions.
Thus, in our work, we use mascots, tablet, lamp, and speakers as a representation of inanimate
objects called “things”, which can interact with each other and send information over the local network.

Also, this paper provides key technology of IoT such as Radio Frequency Identification,
Electronic Product Code, ZigBee, etc.
From the technical point of view, RFID is primarily relevant to the unique identification of
a “thing” to communicate with other objects.
Moreover, ZigBee is widely used, short-range, low-rate wireless network technology, so in our
system, the communication between mascot and lamp is built with the help of the Zigbee Lighting protocol.
Additionally, an inexpensive radio technology Bluetooth Low Energy is also very useful
for our research for proximity sensing.
In addition to these technologies which are considered as a pillar for the communication between
objects, the more detailed description of their usage can be found in the implementation chapter (see chapter 4).

An example of research work in the area of IoT may be “Explorations on Reciprocal
Interplay in Things Ecology”~\cite{chung2018explorations} where the authors are trying to
stimulate scientists to a more detailed discussion about the design qualities of the IoT devices.
For that, Chung et al conducted the HiddenLocal workshop (HWL) to explore and design
IoT systems, where they take into account reciprocal interplay believing that it makes the
design of IoT systems more dynamic.
As a starting point, authors show 7 perceptual qualities as follows: focus the senses;
show explorative behavior;
subtleness of movement;
react to the external event;
recognize explorative behavior subject;
reflex contextual noise;
remember and anticipate perception over time.
Authors believe that these perceptual qualities are a good approach for designed explorative
features of devices and therefore for the things-to-things interaction.

\section{Social Internet of Things}
\label{sec:Social Internet of Things}
According to the “The Internet of Things: A survey” paper ~\cite{atzori2010internet},
there are many research issues related to the IoT that require further
research and need to be addressed.
One of them is that people still cannot be sure about the privacy of the transferred data
through IoT technologies.
Another issue is network navigability which must ensure that the discovery of objects can be
performed effectively and a better reaction to the state changes of objects.
Atzori et al ~\cite{atzori2011siot} formalized a new paradigm of Social Internet of Things (SIoT)
where the interaction among smart objects is based on the notion of Social
Relationship of things rather than their owners.
Thus, the application of this concept to the IoT can lead to the improvement
of the network navigability and scalability.
The architectural model of SIoT describes the establishment of the social relationships
among objects in a fashion that is relatively similar to the human social network relationship.

Applying a new paradigm to the IoT concept can lead to the following advantages:
\begin{itemize}
  \item Establishing the level of trustworthiness by leveraging relationship types and by
        supporting services usable among things that are “friends”.
  \item Improvements in network navigability.
  \item A guarantee of higher scalability and efficiency~\cite{atzori2012social}.
\end{itemize}

By integrating social networking concepts into the Internet of Things, intelligent things
establish a connection with other peers autonomously by exploiting things' social relationships.
An exemplary connection between smart things that we also refer to as "social things" in our
study may be mascot-mascot, mascot-tablet, mascot-lamp, and mascot-speakers interactions.
The application of the SIoT concept will help to accomplish complex tasks such as changing
object behavior according to the given information.
Therefore, with the help of the advantages provided by social networking principles,
the IoT evolves into the concept of the SIoT\@.
Thus, social relationships can be established among
the devices in order to advertise information about their current state and provide services to their peers.

\section{Autonomous interaction of things}
\label{sec:Autonomous interaction of things}
Most devices using the concept of IoT are designed to involve the user in the process
where the user's actions trigger certain functions of a system in order to effects the behavior of objects.
This design contradicts the concept of a fully automated system where objects
can cooperate with each other beyond the control of a human.
The following paper ~\cite{okada2016autonomous}, which is an inspiration for our work,
introduces the design methodology to achieve a more autonomous system.
The authors applied the concept of SIoT and consider objects as living beings that are
able to communicate with others and exchange information autonomously.
This approach allows objects to have their own social circle similar to a human social network.
This broadcast information calls certain functions that affect the behavior of objects,
thus, allowing objects to be aware of the status of other objects and the surrounding environment.

The concept of Social Things which is also essential for our system helps the objects to know:
\begin{itemize}
    \item Their goals.
    \item What to do with the received information.
    \item What actions need to be taken to achieve these goals.
\end{itemize}
In our work, goals and the combination of actions that will be triggered
depending on the received information, are all predefined.

As a case study, the authors developed a system with two devices.
One of them is the mascot - is a small keychain of three colors:
red, green and blue presented in the form of a personal object that the user can carry with him everywhere.
Another device is a bench with built-in lamps presented as a more static device for public use.
In our system, the device that is called mascot is a user's smartphone and as an
analogy of the bench, we use the term lamp.
The description of all further devices used in our system is given
in Section~\ref{sec:Identifying case studies and actions}.

Besides, in the system described in the paper~\cite{okada2016autonomous}, two scenarios are considered:
\begin{itemize}
  \item \textbf{Mascot - mascot} interaction, as one mascot approaches another, they both start to blink where the
        intensity of blink depends on the distance objects are from each other.
  \item \textbf{Mascot - bench} interaction, as mascot goes close to the bench, the lights
        start to change their colors based on the color of the mascot that is approaching.
\end{itemize}

The biggest contribution of this paper was to introduce the autonomously cooperative
system where mascot and bench represented as private and public things.
Moreover, authors also considered proximity-based cooperation: devices blink more often when
approaching closer than 30 cm and blink with less intensity when approaching more than 150 cm.
By using the concept presented in this study, namely, autonomous interaction between objects-things
achieved with the help of the SIoT concept, we are planning to expand the system by adding more objects.
Afterward, we are going to apply the theory of proxemics and the concept of the personality
traits which will be covered in the following sections.
In addition to the two categories that authors described in their paper i.e.\ private and public "things"
presented by two objects (mascot and bench, respectively), in Section~\ref{sec:The Theory of Proxemics}
we are planning to look at more detailed divisions.

\section{The Theory of Proxemics}
\label{sec:The Theory of Proxemics}
Edward Hall ~\cite{hall1966hidden,hall1963system} conceptualized the idea of a
personal space bubble by creating a whole system of notation to understand
and record how people navigate shared space.
He correlated physical distance to social distance.
According to these papers, Hall identified four distances which are measured horizontally:
\begin{itemize}
  \item \textbf{Intimate distance} which varies from 0 to 45 cm is a distance used for romantic partners and family members.
  \item \textbf{Personal distance} varies from 46 to 121 cm is a space bubble which allows your extended family
        members and close friends to enter this zone.
  \item \textbf{Social distance} varies from 122 to 369 cm is often used for acquaintances and colloquies
  \item \textbf{Public distance}, having a range of 370 cm and more, is often used in public speaking
        situation and with strangers, you want to maintain your distance from.
\end{itemize}

He also analyzed vertical distances, for example, the difference in vertical distance
between people can reflect the degree of dominance.
However, in our study, we focus only on horizontal distances.

Nowadays, there are many studies in which Proxemics has been used to design interactions.
For example, Jo Vermeulen et al in their work ~\cite{vermeulen2015proxemic} used zones to
interact with vertical interactive displays where they suggested floor display as an auxiliary device.
The contributions of using the secondary display are the following:
\begin{itemize}
    \item Provides peripheral information about the tracking status of a user.
    \item Shows interaction zones.
    \item Invites the user to interact with the main display.
    \item Suggests possible interaction steps.
\end{itemize}
This kind of floor visualization with continuous feedback about proximity gives the user
more control over their interaction with the system.

Another example system using Proxemics is the Remote Controls system introduced by
Ledo et.al in their ~\cite{ledo2015proxemic} paper.
Remote control devices were created in such a way that people could
control appliances from a certain distance.
However, with the increase in the number of home appliances, the number of remote controls also increasing.
For this purpose, the universal remotes have been proposed providing a one-remote-to-many-appliances solution.
Unfortunately, this design has setup issues and poorly adaptable interface.
The authors of this paper presented the proxemic-aware controls that utilize the spatial relationship between
mobile devices owned by a user and appliances surrounding it.
With this system, a user can discover and select the devices within large ecologies of appliances,
view their current status, and control their features.
Moreover, as a user moves closer or farther to a particular device, the interface adjusts accordingly.
For example, in the initial state, the tablet screen visualizes icons representing the
location of appliances at the edge of the screen, these icons are dynamically updated as he moves.
Through spatial interactions, people can leverage mobile devices to discover and select appliances.
This allows for situated interaction that balances simple and flexible control while seamlessly
transitioning between different control interfaces.
Ubicomp, which they use as short for Ubiquitous Computing, is a starting point for
developing a new type of remote control interface within our increasingly complex world.

Also, Ballendat, Nicolai Marquardt, and Saul Greenberg in their
work~\cite{ballendat2010proxemic} introduce proxemic-aware interactive media player system,
where they consider information regarding nearby people and devices to mediate the interaction.
They cover a small space Ubicomp environment considering the relationships of people to devices;
devices to devices;
and non-digital objects to people and devices.
The system reacts to a person’s presence, distance, and orientation regarding the display.
Proxemic interaction also considers a person’s relationship with nearby objects.
The authors propose different cases, for example, the video displaying on the screen pauses
when a person is having a phone conversation or when he picks a magazine to read it.
Another case is when a person enters the room, the screen shows a video title as additional information for him.
Moreover, the video is paused when both people face away from the screen to start a conversation with each other.
Furthermore, the system turns off when everyone has left the room.
However, the authors also emphasize that one of the biggest unsolved
problems in this area might be how the system can respond to the received information about
proxemics because sometimes the devices can make a mistake by taking a certain action.
Despite all these problems, the authors, as well as we believe that proxemics will
become an important factor in the embodiment of the interaction between social objects
where they can meet the social expectations of people.

In our research, we are planning to extend the autonomous system, which was
described in Section~\ref{sec:Autonomous interaction of things}, by categorizing devices
according to the theory of Proxemics using only horizontal measurements.
Our goal is to cover all four categories of Proxemics which are represented by four
artifacts such as mascot, tablet, lamp, and speakers, thereby, constituting four case studies.
Thus, each of these devices is located at a certain distance from each other representing
the relationships between them.
These relationships will help us to conceptualize their interactions, come up with case studies and possible actions.

\section{Interaction design for SIoT}
\label{sec:Interaction design for SIoT}

The following paper ~\cite{soro2018social} motivated us to apply a concept of
personality in the context of social devices.
An example use of personality as a method to design an interactive object’s behavior
was proposed in “Designing the Behaviour of Interactive Objects” paper~\cite{spadafora2016designing}.
The author concluded that to design a more stable and understandable
for a user behavior of a device, it is necessary to add an inner logic to which we can refer.
Marco et al proposed to apply the concept of metaphor, which represents human stereotypes
of personality to visualize the inner logic.
Their system was based on a Big Five Personality Traits model, thus,
by assigning these personalities to objects, users could describe its behavior more easily.
The authors believe that stereotypes and metaphors are simplified descriptions of being and behavior,
therefore, making it an ideal method for displaying the sustainable behavior of a smart object.
During the research, they used the robotic sofa as a case study and tried to analyze how
users perceive the consistency of its behavior.
The use of the personality model in the device design process, helps a user to create
a mental model of how an autonomous sofa-bot will act in the future.

Having a system, where tablet, lamp, and speakers are considered as static objects,
whereas mascots are dynamic, we can apply the Personality Model.
Since only mascots are a major factor affecting the environment
(for example, when mascot come close to the lamp, it changes the light color), we decided to
assign a personality to dynamic objects, particularly, to each mascot.

This approach may help the user to better understand the mascot's behavior.
Knowing which goals and intentions this object follows, may help users to understand the
behavior and the reason for certain decisions of a mascot.
Thus, the device with an assigned personality can help users who know the definition
of that personality to understand the system behavior at least in an intuitive way.
This concept may give the system a more understandable and consistent behavior
and to the user a better awareness of object functionality.
In the following subsection, we will describe the Personality Model in more detail.

\section{Definition of personality traits}
\label{sec:Definition of Personality Traits}

The personality is important for human relationships, so we assume that it also may be important for device relationships.
To assign a personality to each mascot, we first need to define personality in the context of social devices.
We can try to intuitively explain the meaning of a person’s personality trait,
however, it is hard to apply it in the context of SIoT\@.
For that, we need a generally accepted model for the description of each personality which is a
Big Five Personality Traits aka OCEAN Model\@.
We expect that describing personality will help to define goals and
more targeted actions which in turn will lead the system to more stable behavior.
The following book ~\cite{matthews2003personality} gives a good introduction to the
personality types describing possible existing personality models.
One of the models that can be used was introduced by Costa and McCrae’s five-factor
model which is also known as Big Five Personality Traits and the OCEAN model.
Moreover, their concept formed a basis for the widely used NEO-Personality
Inventory-Revised (NEO-PI-R) measurement scale.

The OCEAN model consists of the following features: \textbf{O}penness to experience,
\textbf{C}onscientiousness, \textbf{E}xtroversion, \textbf{A}greeableness, and \textbf{N}euroticism.
The authors lists the facets associated with each of these five domains:

\begin{itemize}\label{table:opennessFacets}
  \item  \textbf{Openness to experience}: creativity, innovative quality;
          quick receptivity to new and abstract ideas, high intelligence and openness to novelty;
  \item  \textbf{Conscientiousness}: organized, well-prepared, discipline, likes planned action more than spontaneity, more focused.
  \item  \textbf{Extroversion}: energetic, assertive personality, like to be the center of attention,
          like to dominate, feel comfortable around people;
  \item  \textbf{Agreeableness}: friendliness, compassion for other people, interested in people,
          sympathize with the feelings of others, soft-hearted;
  \item  \textbf{Neuroticism}:  irritability, more hostile towards others, most often feel anxiety
          when they are surrounded by others, frequent mood swings, emotionally unstable;
  \end{itemize}

The dynamic objects are assigned with a set of personality traits each of which
is described in the above-mentioned list.
Although people usually have a combination of these five traits,
we are going to consider only extreme cases and take into account each personality independently.
