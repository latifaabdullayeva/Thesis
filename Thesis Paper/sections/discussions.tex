\chapter{Discussions}
\label{ch:discussions}
This chapter covers the discussion of the results obtained from the statistical tests.
In this study, the behavior of a mascot is conceptualized into four interaction types which are referred to as case-studies.
Sections~\ref{sec:mascot-speakers-interaction}, ~\ref{sec:mascot-lamp-interaction},
~\ref{sec:mascot-mascot-interaction}, and ~\ref{sec:mascot-tablet-interaction} discuss each interaction type for the
within the personality and the within condition studies.

\section{Mascot-speakers interaction}
\label{sec:mascot-speakers-interaction}
The empirical evidence for mascot-speakers case-study shows that the type of music influences the participants'
interpretation of the personality trait.
The mascot triggering sophisticated, contemporary, or unpretentious song can be associated
with a specific personality trait. \\

\textbf{The discussion of the first study.}\par
On the one hand, the interaction between mascot and contemporary was interpreted
to be more extravert which is characterized as being more friendly,
sociable, cheerful, and so on.
On the other hand, this type of interaction also was rated very high on conveying
neuroticism personality trait which is described as being more emotionally unstable,
angry, vulnerable, depressive, and so on.
The mascot triggering sophisticated music was perceived to be more open to new experiences.
Such songs as classical, jazz and a contemporary adult were interpreted as more artistic intelligent,
thoughtful, and other facets representing openness personality trait (see Table~\ref{table:personality}).
Agreeableness and conscientiousness personality traits were associated with sophisticated and
unpretentious compared to contemporary music.
However, this study does not provide statistical evidence for these two personality traits to be
conveyed by a specific music type.

\textbf{The discussion of the second study.}\par
The within music condition study revealed that mascot interacting with sophisticated music
was perceived to be more open to new experiences.
In addition, the first study associated contemporary with both extraversion and neuroticism personality traits.
By comparing the ratings of all personality traits within contemporary music, the second study revealed that this
music condition conveys an extraversion personality trait most.


\section{Mascot-lamp interaction}
\label{sec:mascot-lamp-interaction}
The study shows that the change in color of the lamp has an impact on the way
how participants interpret the personality of a mascot.
This shows that the lamp that mascot interacts with was not only interpreted as an
artificial light source but also gave clues about mascot's personality trait.\\

\textbf{The discussion of the first study.}\par
Participants associated mascot triggering blood-red lighting with neuroticism personality trait
which is characterized as to be highly anxious, angry, depressive, self-conscientious, impulsive, and vulnerable (see Table~\ref{table:personality}).
Extraversion personality was rated very high to be conveyed with yellow lighting only compared
to turquoise, blood-red, and pink.
With agreeableness and conscientiousness, participants rated mascots interacting with pink, yellow,
turquoise lighting to be more associated with these personality traits.
However, for all personality traits except neuroticism, it is hard to distinguish
a specific color that conveys these personality traits.\\

\textbf{The discussion of the second study.}\par
The mascot triggering the blood-red lighting is interpreted to be more neurotic i.e. such as an anxious, highly depressive,
angry, vulnerable, and that has an immoderate behavior (see Table~\ref{table:personality}).
The second study confirms that how the personality of a mascot was measured, except
for neuroticism, all other personality traits cannot be conveyed by a single color.
For example, both turquoise and pink lights convey agreeableness and
conscientiousness personality traits.
Also, the orange color was interpreted to be more extravert and open to new experiences.\\


\section{Mascot-mascot interaction}
\label{sec:mascot-mascot-interaction}
The statistical analysis showed that the levels of vibration have a significant effect on the
measurement of mascot's personality.\\

\textbf{The discussion of the first study.}\par
Participants experiencing 500 milliseconds vibration duration interpreted the approaching mascot as more
energetic, assertive, forceful, energetic, friendly, sociable, cheerful which
constitutes extraversion personality trait (see Table~\ref{table:personality}).
The mascot interacting with level-2 and level-3 was perceived to be more agreeable which
is described as being modest, cooperative, trustworthy, and so on.
Mascot interacting with level-3 and level-4 vibration are interpreted as conscientiousness.
The participants experiencing 100-millisecond vibration aka level-1 interpreted
approaching mascot as neurotic which is characterized by being
anxious, angry, depressive, and so on (see Table~\ref{table:personality}).\\

\textbf{The discussion of the second study.}\par
Participants experiencing the vibration with the shortest duration
perceived approaching mascot (i.e.\ the mascot that triggered their device) as neurotic.
The mascot causing 200 milliseconds per time interpreted as agreeable.
The mascot making participant's phone to vibrate with the duration of 500
milliseconds conveys extraversion personality trait.\\

\section{Mascot-tablet interaction}
\label{sec:mascot-tablet-interaction}
The empirical evidence revealed that, overall, the change of background
color effects which personality trait will be perceived by this interaction.\\

\textbf{The discussion of the first study.}\par
For the within personality study, turquoise which is distinguished from all other
colors represents conscientiousness personality trait.
Also, blood-red color is associated with neuroticism personality that is characterized by
having immoderate behavior, being aggressive, depressive, and so on (see Table~\ref{table:personality}). 
Openness personality is best conveyed by a yellow screen color.\\

\textbf{The discussion of the second study.}\par
The analysis of the within color condition shows that the yellow background portrays an openness
personality trait which makes the results from the first study consistent with the second study.
Also, a blood-red screen is significantly associated with neuroticism personality.
The first study revealed a conscientiousness personality being conveyed by turquoise color.
However, when all personality traits were compared within turquoise color, the ratings
of conscientiousness personality traits did not distinguish from the scores given for
agreeableness personality.
Thus, according to the second study, turquoise color failed to be associated only with conscientiousness personality.

\section{Overview of the discussion}
\label{sec:overview-of-the-discussion}
During this study, for each interaction type, there are actions such as music play, vibration, lighting, and
screen color change that can be associated with specific personality traits.
The representation of all types of behavior and associated personality traits are the following:

\begin{itemize}
    \renewcommand{\labelitemi}{$\Rightarrow$}
    \item \textbf{Music type in mascot-speakers interaction.}
    \begin{labeling}{alligator}
        \item [Extraversion:] contemporary songs.
        \item [Openness:] sophisticated songs.
    \end{labeling}

    \item \textbf{Lighting color in mascot-lamp interaction.}
    \begin{labeling}{alligator}
        \item [Neuroticism:] blood-red.
    \end{labeling}

    \item \textbf{Vibration level in mascot-mascot interaction.}
    \begin{labeling}{alligator}
        \item [Neuroticism:] level-1 (100 ms).
        \item [Agreeableness:] level-2 (200 ms).
        \item [Extraversion:] level-5 (500 ms).
    \end{labeling}

    \item \textbf{Screen color in mascot-tablet interaction.}
    \begin{labeling}{alligator}
        \item [Neuroticism:] blood-red.
        \item [Openness:] yellow.
    \end{labeling}
\end{itemize}