\chapter{Conclusion}
\label{ch:conclusion}
Section 8.1 of the final chapter gives a conclusive overview of the results achieved during this study.
Section 8.2 describes the contribution, Section 8.3 provides certain limitations,
and Section 8.4 covers the ideas explored in this project can be continued further.

\section{Overview}
\label{sec:overview}
In this study, both psychological and user‐experience approaches help us to better understand thing-to-thing interaction.
The interaction between a mascot and other inanimate devices gave a
descriptive clue about the personality of a social device.
In this study, we explore how people interpret the interaction between user and social devices.
The preset behaviors such as playing music, vibrating with a certain duration, changing lighting colors,
and altering screen color were investigated to find an association with the concept of personality trait.

\subsection{Overview of the results}
\label{subsec:overview-of-the-results}
In the first study, we observe the personality trait and the variation of all
conditions within each personality individually.
In our second study, we take a closer look at each condition and the variation of
all personality traits within a specific condition - specific color, music type, and vibration level.
The second study is supplementary for the first study.
Thus, the second study results in a specific personality trait that is conveyed most by this condition.

The following guideline to associate predefined actions with a specific personality trait is introduced:

\begin{itemize}
    \item For the \textbf{Mascot-speakers interaction}, in order to convey an extraversion personality trait,
    contemporary music will be the best choice.
    For the mascot that attributes an openness personality trait,
    the good choice will fall on sophisticated music.
    \item For \textbf{Mascot-lamp interaction}, if the mascot triggers blood-red lighting as a
    representation of its behavior, it will convey the neuroticism personality trait.
    \item For \textbf{Mascot-mascot interaction}, the extraversion mascot can be presented by
    showing the highest level of vibration, namely level-5 with 500 milliseconds duration.
    The best representation of an agreeableness personality trait will be level-2 vibrating 200 milliseconds per time.
    When the neuroticism is chosen, the behavior that conveys this personality trait most is the vibration level-1.
    \item For \textbf{Mascot-tablet interaction}, in order to convey the openness personality trait,
    the most associated behavior is choosing a yellow screen.
    For neuroticism personality trait, the blood-red is a good and distinctive choice.
\end{itemize}

In addition, a social device interacting with the same color in the various environment was interpreted differently.
For example, the yellow screen conveyed openness for mascot-tablet interaction, whereas, for mascot-lamp,
it failed to convey any personality trait.
One reason might be that the lamp emitting yellow light can be perceived
as the typical color you get from incandescent bulbs.
Meanwhile, the yellow screen color is perceived as a more vivid color framed on a screen.
The same observation is made for blood-red color which conveys a neurotic personality trait for both interaction types.
However, the mascot triggering the blood-red lighting is perceived as
more aggressive than the one triggering the same color in the tablet.
By illuminating the whole room with blood-red color people may
get a more negative impression about mascot’s personality than seeing the same color in a tablet-size screen.
The above-mentioned explanations are assumptions and can be investigated in further studies.

\section{Contributions and findings}
\label{sec:contributions-and-findings}
The main contribution is achieving cooperation among mascots and other interactive
objects in a system where each mascot has a unique personality trait.
This can, in turn, serve as a contribution to designing social devices in the SIoT environment.
Thus, the system that produces four types of interactions with predefined actions was implemented.

We also empirically investigated how people interpret the behavior, particularly such
signals as lighting color change, music play, vibration, and screen color alternation.
The important finding was the relationship between personality traits and behavior of interactive objects.
Although this relationship is not as clear and vivid as we observe in human-human interaction,
it is a good insight that personality traits and behavior are interconnected concepts even for inanimate objects.

Finally, we used personality based on the Big Five Personality Trait Model as an interaction
between people and social devices, where personality was conveyed by the predefined actions.
This shed a light on using the personality model as a tool to predict and influence automated behaviors
in the context of the Internet of Things environment.

\section{Limitations}
\label{sec:limitations}
The study was based on relatively small (N=25) and homogeneous
(e.g.\ having 70\% of participants from formal sciences and age mean of 26) samples which
limited the power of analyses and make it difficult to extrapolate findings to a general population.
However, the experiment design was counterbalanced in terms of other important
characteristics such as gender, participants’ overall music preferences for mascot-speakers
interaction, and the order in which participants watched the videos.

Another limitation could be that the study was held in a laboratory setting where
participants only had a limited time to assess the personalities of mascots.
On the one hand, there is a possibility that spending more time (i.e.\ hours or days)
with social devices could affect the participant’s opinions on the measurements of the personality trait of devices.
On the other hand, during these experiments participants reflected their very first impressions
while seeing the interaction between a user and devices and their initial reaction while measuring the
personality traits of mascots based on these interactions.

Also, during the study, some signals especially some colors in the mascot-lamp interaction was
hard to associate with specific personality traits.
However, this study gave us insight that personality traits and the behavior
of social devices are two interrelated concepts.
In addition, for the preset actions and signals that were associated with personality traits,
the guideline was introduced.

The limitation regarding the implementation was to have a centralized system that gives
control to the server instead of the social devices themselves.

\section{Future work}
\label{sec:future-work}
From the implementation perspective, a distributed system for communication between
social devices can be designed.
Thus, the realization of a protocol that helps all devices in a system to come to a
common agreement can also be performed.

Another idea might be improving mascot-tablet interaction in addition to screen colors
some other functionalities can be added.
For example, when a mascot approaches a tablet, it can show some favorite pictures
to give more information about the itself.
In fact, the design of such a system should also take into account security aspects for not leaking personal photos.

Also, the interaction between social devices can be used as a trigger or
motivation for people to socialize and communicate more.
The behavior of mascots such as a light change or a music play or vibration can be an icebreaker for people to communicate with each other.