\section*{Abstract}
In a Social Internet of Things environment, smart devices autonomously
communicate with each other by establishing their own network.
While social device interaction increases scalability and enhances network navigability,
decisions made by smart devices are not always clear and may not be in line with a user's expectation.
Our work aims to find an approach that will bring awareness of automated behaviors
while taking the user's context into account.

In our prototype, a user's smartphone becomes their personal Mascot able to autonomously
connect to four other devices, i.e., another Mascot, a lamp, a tablet, and speakers.
While proximity is used to initiate communication between devices, the resulting
behavior is influenced by a predefined "personality" of the user's Mascot.

In a user study, we explored different automated behaviors and investigated how users
perceived the associated personality according to the Big Five Personality Trait model.
Our results indicate that different types of actions, such as playing certain types of music,
vibrating at a certain level, changing lighting, and altering screen color, are interpreted as certain personalities.
This opens the path to utilizing personality traits as tools to predict and influence
automated behaviors in an Internet of Things environment.