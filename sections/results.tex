\chapter{Results}
\label{Results}
This chapter presents the statistical analysis of our study where the results
for each use case, namely, Mascot-Lamp, Mascot-Mascot, Mascot-Tablet, and
Mascot-Speakers interactions are reported in each section separately.
In each section, there are two subsections where we report statistical tests for two
studies: within personality trait and within condition such as lighting color, music
category, vibration level or screen color.
In our first study, we focus on each personality trait separately
and the difference between each state of conditions within that personality trait.
In the second study, we focus on each state of condition separately
and the difference between five personality traits within that condition state.

\section{Analysis of Mascot-Lamp interaction}
\label{M-L}
This section describes each personality trait that Mascot was assigned in terms of
the effect of the lighting color in evaluating them.
The factors that we compare for Mascot-Lamp interaction are orange, turquoise,
yellow, blood-red and pink lighting colors.
Since the data that we gathered are ordinal and we do not assume that the outcome
will be normally distributed, we focused on non-parametric tests for all case studies.
Moreover, since each case-study consists of more than two compared groups
(i.e in case of Mascot-Lamp interaction we compare five colors with each other)
and compare data against within-subject factor (i.e each participant tested all five conditions),
we decided to use Friedman test followed by the Wilcoxon Signed-rank test.
In both studies, we analyze the effect of the lighting color on how the mascot’s personality is measured.
The first study analyzes this effect within each personality trait, particularly, we consider each
personality trait and the effects of each color on participant's measurements of Mascot's personality.
The second study considers this effect within each color condition,
namely how each personality trait is assessed differently within one color condition.
For each study Wilcoxon test compare 10 groups with each other which makes 20 compared groups in total.
Since, in our study, we have a large number of statistical tests, some of the results may have p<0.05 purely by chance.
Thus, in order to control family wise error rate, we use Bonferroni correction which will
divide all p-values in 20 (the number of compared groups for both studies).
Finally, at the end of each subsections, we show a graphical display of the results from each study.
Subsection~\ref{Study1(M-L)} describes the results for within personality study and
subsection~\ref{Study2(M-L)} for within lighting color study.

\subsection{Analysis of within personality trait study}
\label{Study1(M-L)}
In the first study, we test the effect of all predefined lighting colors on the measurements of each personality trait.
Friedman tests reported in Table~\ref{table:friedmanML1} reveal a significant impact of lighting colors on the perception
of the personality trait of mascots with p < 0.01.

\begin{table}
\renewcommand{\arraystretch}{1.2}
\begin{center}
\begin{tabular}{p{0.05\textwidth}|
			p{0.025\textwidth}|p{0.025\textwidth}|p{0.025\textwidth}|p{0.025\textwidth}|p{0.025\textwidth}||
			p{0.025\textwidth}|p{0.025\textwidth}|p{0.025\textwidth}|p{0.025\textwidth}|p{0.025\textwidth}||
			p{0.025\textwidth}|p{0.025\textwidth}|p{0.025\textwidth}|p{0.025\textwidth}|p{0.025\textwidth}|}
\cline{2-16}
  & \multicolumn{5}{c||}{\textbf{Extraversion}} & \multicolumn{5}{c||}{\textbf{Agreeableness}} & \multicolumn{5}{c|}{\textbf{Conscientiousness}} \\
\cline{2-16}
 			& Y & O & T & B & P 			& Y & O & T & B & P  	 	& Y & O & T & B & P     \\
\cline{2-16}
\textbf{Min}  	& 2.3 & 2.3 & 1.8 & 1.0 & 1.7 		& B1 & B2 & B3 & B4 & B5  	& C1 & C2 & C3 & C4 & C5  \\

\textbf{Med} 	& 3.7 & 3.0 & 2.7 & 1.7 & 2.8 		& B1 & B2 & B3 & B4 & B5  	& C1 & C2 & C3 & C4 & C5  \\

\textbf{Max}	& 4.8 & 4.8 & 4.2 & 4.5 & 3.7 		& B1 & B2 & B3 & B4 & B5  	& C1 & C2 & C3 & C4 & C5 \\
\cline{2-16}

\cline{2-11}
&  \multicolumn{5}{|c||}{\textbf{Neuroticism}} & \multicolumn{5}{|c||}{\textbf{Openness}} \\
\cline{2-11}
				& Y & O & T & B & P 			& Y & O & T & B & P    		\\
\cline{2-11}
	\textbf{Min} 	& D1 & D2 & D3 & D4 & D5 		& E1 & E2 & E3 & E4 & E5 	\\

	\textbf{Med}  & D1 & D2 & D3 & D4 & D5 		& E1 & E2 & E3 & E4 & E5 	\\

	\textbf{Max}  	& D1 & D2 & D3 & D4 & D5 		& E1 & E2 & E3 & E4 & E5  	\\
\cline{2-11}

\end{tabular}
\end{center}
\caption{Some Caption Y is yellow, O is orange \ldots}
\label{table:e-c}
\end{table}


\par \textbf{Extraversion.}
According to Table~\ref{table:friedmanML1}, lighting colors significantly
influenced the measurements of Extraversion personality.
Figure (--M-L-study1) reports where exactly this effect is concentrated.
Based on Wilcoxon tests, there are six groups of colors affecting the
measurements of Extraversion personality with p<0.01.
In addition, yellow showed significant difference in rating extraversion
comparing to turquoise, blood-red and pink lighting colors.
Participants rated Mascot's personality trait based on six facets of five
personality trait which makes in total 30 facets.
They rated the mascot that triggered yellow lighting color high on being
friendly, gregarious, assertive, energetic, excitement seeking and
cheerful in comparison to mascot that triggered blood-red, turquoise and pink colors.
All above mentioned facets represent extraversion personality trait (see Table --facets/traits).
Thus, participants rated the mascot interacting with yellow lighting to convey extraversion personality.
In contrast, the mascot interaction with blood-red lighting was
rated very low on being extravert (see Figure--M-L-study1).
This is also reported in Table---median, the blood-red color having
the lowest median (Med = 1.7, Max = , Min = ) and yellow having the highest
value (M = 3.7, Max = , Min = ).

\par \textbf{Agreeableness.}
There is a significant impact of lighting colors and the participants'
measurements of agreeableness personality with p<0.01 (see Table~\ref{table:friedmanML1})
According to Figure(--), mascot triggered blood-red and orange lighting
were rated very low on being agreeable (padj<0.01).
Moreover, Table--median shows that in comparison to all other colors,
blood-red color has the smallest median values with Med = 2.0, Min = , Max = .
The median values, in descending order, for pink, turquoise and yellow lights are
approximately similar (M = 4.0, M = 3.5 and M = 3.7).

\par \textbf{Conscientiousness.}
Friedman test shows statistically significant effect of all predefined lighting colors
on the ratings of Conscientiousness personality (see Table~\ref{table:friedmanML1}).
Wilcoxon tests show that effect is noticeable when we compare mascot that trigger
blood-red and orange with ones that triggered turquoise, yellow, pink colors (see Figure--).
In fact, the mascot interacting with blood-red and orange lighting were assessed
as being very low on conscientious personality trait.
Table--median indicates that blood-red has a lowest median (M = 2.2), whereas turquoise, pink and
yellow have relatively similar high medians (M = 3.7, M = 3.5, M = 3.5 respectively in descending order).
The latest values shows that the mascot triggering these colors were assessed high on having
orderly, dutiful, disciplined and other facets that constitute conscientiousness personality.

\par \textbf{Neuroticism.}
Overall, there is an impact of predefined colors on the rating's of Neuroticism
personality with p<0.01 reported in Table~\ref{table:friedmanML1}.
Blood-red showed a significant difference in rating Neuroticism comparing
to all other colors with padj<0.01 after Bonferroni correction (see Figure--).
Moreover, blood-red presented the highest median value (Med = , Max = , Min = ) (see Table--median).

\par \textbf{Openness.}
There is a significant difference of all colors within openness personality
with p<0.01 (see Table~\ref{table:friedmanML1}).
Moreover, the main differences are concentrated between yellow and pink, yellow and blood-red,
blood-red and orange with padj<0.01 (see Figure--).
Table--median shows the similarity of the median values for all colors concentrating around neutral attitude for mascot
being measured as openness which is represented by median close to 3 (see Appendix A.1).

Figure indicates only significant comparisons with p<0.05.
We report p-values adjusted after Bonferroni correction for more
detailed information see Table--wilcoxon in Appendix.


\subsection{Analysis of within lighting color study}
\label{Study2(M-L)}
In both studies, we analyze the effect of the lighting color on how the mascot’s personality is measured.
The first study analyzes this effect within each personality trait separately, particularly, we consider each
personality trait and the various effects of each color within this personality. The second study considers this
effect within each color condition separately, namely how each personality trait is assessed differently within one color condition.
For the second study, Friedman test reveals a significant difference of all personalities within each color
condition with p < 0.01 (see Table~\ref{table:friedmanML2}). Pink and blood-red conditions have the most significant
effects on the measurements of the personality traits
(Z=60,082, p< 0.01 and Z=45.475, p< 0.01 respectively). 

\begin{longtable}{ |p{1.7cm}| p{1cm}|p{0.5cm}|p{1.7cm}| }
\captionsetup{width=13.5cm}
\caption{The results from Friedman test for all Five Personality traits in case of Mascot-Lamp interaction}
\label{table:friedmanML2} \\
\hline
 \multicolumn{1}{| c}{\textbf{Personality trait }} 
  & \multicolumn{1}{| c}{\textbf{$\chi^2$}}  
  & \multicolumn{1}{| c}{\textbf{df}} 
  & \multicolumn{1}{| c |}{\textbf{p}}  \\
\hline 
\endfirsthead
\multicolumn{4}{c}%
{\tablename\ \thetable\ -- \textit{Continued from previous page}} \\
\hline
 \multicolumn{1}{| c}{\textbf{Personality trait }} 
  & \multicolumn{1}{| c}{\textbf{$\chi^2$}}  
  & \multicolumn{1}{| c}{\textbf{df}} 
  & \multicolumn{1}{| c |}{\textbf{p}}  \\
\hline
\endhead
\hline \multicolumn{4}{r}{\textit{Continued on next page}} \\
\endfoot
\hline
\endlastfoot
Yellow		&23.566	&4	&p<0.01 \\
Orange		&38.178	&4	&p<0.01\\
Turquoise		&37.123	&4	&p<0.01 \\
Blood-red		&45.475	&4	&p<0.01 \\
Pink			&60.082	&4	&p<0.01 \\
 \hline 
\end{longtable}

\par \textbf{Yellow.} According to WIlcoxon test using Bonferroni correction, there is a significant negative
correlation between yellow lighting color and neurotic personality trait (padj<0.01). The effect sizes of each color
condition group show that when participants were assessing the personality traits based on a yellow color, 79\% of
scores given for extraversion, agreeableness, conscientiousness, and openness were higher than all scores given for
neurotic personality (see Table~\ref{table:wilcoxML2}). In addition, the similarity of the median values for all
personality traits except neuroticism (M = 3.7, M = 3.7, M = 3.5, M = 3.5) reveals a small effect of yellow color
on these four personality traits (see Appendix A.2). Thus, for yellow color only negative correlation is statistically substantial.
\par \textbf{Orange.} Statistical tests report a positive relationship between extraversion and openness personality
traits, and orange lighting color with p < 0.05 (see Table~\ref{table:wilcoxML2}). Thus, when participants were
measuring the personality traits based on orange lighting color, they gave high scores on mascot being perceived as
extravert and openness. The median values and the Bonferroni correction for orange color did not show any significant
differences between extraversion (M = 3.0, Q3 = 4.3) and openness (M = 3.0, Q3 = 3.7) (see Appendix A.2 and Table~\ref{table:wilcoxML2}).
\par \textbf{Turquoise.} Wilcoxon test using Bonferroni adjustment shows that when the turquoise lighting is displayed,
the agreeableness and conscientiousness personality traits are substantially distinguishable from other personality traits.
Thus there is a positive correlation between these personality traits and turquoise color with p < 0.05
(see Table~\ref{table:wilcoxML2}). However, there is conscientiousness and agreeableness are not correlated to each other,
meaning that they both can be measured high when turquoise color is displayed. Because of the p-values, these two
personality traits are not comparable. In Spite of that fact, the median value of conscientiousness (M = 3.7)
is slightly higher than agreeableness (M = 3.5) (see Appendix A.2).
\par \textbf{Blood-red.} There is a significant relationship between blood-red lighting and neuroticism personality
trait with padj<0.01 (see Table~\ref{table:wilcoxML2}). Moreover, according to Figure A.2, there is an excellent separation
of neuroticism boxplot from all other personality traits. In addition, some observations can be classified as potential
outlier for extraversion personality, which may effect the overall dispersion of the boxplot for a larger samples.
Thus, an extraversion might need to merit special attention for a larger survey.
\par \textbf{Pink.} According to the Table~\ref{table:wilcoxML2}, agreeableness and conscientiousness shows a
significant effect on pink lighting color with the p < 0.01 except for conscientiousness and openness group.
Moreover, there is a negative relationship between pink color and neuroticism trait. Based on the medians reported in
Appendix A.2, agreeableness has the highest (M = 4.0) in contrast to the neuroticism which has a lowest value (M = 1.7).


\section{Analysis of Mascot-Speakers interaction}
\label{M-S}
This section includes the Mascot-Speakers use case, where we analyze which effect the music genre has on the assessment
of the Mascot’s personality traits. As we discussed in section 2 and section 3, our choice of the music genre is based
on the MUSIC pattern. For statistical analysis, we examine the effect of each category by distributing genres into
three categories as the following:
\begin{itemize}
  \item Sophisticated:  jazz, classical and contemporary adult
  \item Contemporary: rap, soul, and rap
  \item Unpretentious: pop, rock\&roll / country and bluegrass
\end{itemize}

Subsection 6.2.1 and subsection 6.2.2 describe the statistical results for each study respectively. For the current
case-study, we have also conducted the same tests that we covered in section 6.1.

\subsection{Analysis of within personality trait study}
\label{Study1(M-S)}
In this subsection, we report a statistical analysis of how the impact of the music category varies within each
personality trait. Friedman’s test reported in Table~\ref{table:friedmanMS1} shows that there is a significant
difference in the measurements of the mascot’s personality depending on which song is played. Except for conscientiousness,
all categories reveal their effect (with p < 0.01) on the assessment of each personality trait. However, with p < 0.05,
there is a less than 5\% chance that the music category has no impact on the perception of the mascot’s conscientious traits.

\begin{longtable}{ |p{3cm}| p{1cm}|p{0.5cm}|p{1.7cm}| }
\captionsetup{width=13.5cm}
\caption{The results from Friedman test for all Five Personality traits in case of Mascot-Speakers interaction}
\label{table:friedmanMS1} \\
\hline
 \multicolumn{1}{| c}{\textbf{Personality trait }} 
  & \multicolumn{1}{| c}{\textbf{$\chi^2$}}  
  & \multicolumn{1}{| c}{\textbf{df}} 
  & \multicolumn{1}{| c |}{\textbf{p}}  \\
\hline 
\endfirsthead
\multicolumn{4}{c}%
{\tablename\ \thetable\ -- \textit{Continued from previous page}} \\
\hline
 \multicolumn{1}{| c}{\textbf{Personality trait }} 
  & \multicolumn{1}{| c}{\textbf{$\chi^2$}}  
  & \multicolumn{1}{| c}{\textbf{df}} 
  & \multicolumn{1}{| c |}{\textbf{p}}  \\
\hline
\endhead
\hline \multicolumn{4}{r}{\textit{Continued on next page}} \\
\endfoot
\hline
\endlastfoot
Extraversion		&21.44	&2	&p<0.01 \\
Agreeableness		&29.01	&2	&p<0.01\\
Conscientiousness	&6.4536	&2	&p<0.05\\
Neuroticism		&15.122 	&2	&p<0.01 \\
Openness			&25.838	&2	&p<0.01 \\
 \hline 
\end{longtable}

\par \textbf{Extraversion.} Wilcoxon test using Bonferroni correction showed an extremely significant difference
between Sophisticated and Unpretentious categories with p < 0.001, r=0.864 (see Table~\ref{table:wilcoxMS1}).
According to Appendix A.3, the Unpretentious category has a median higher than for Sophisticated with a value of 3.3
and 3.0 respectively.
Based on Figure A.3 , the samples of Contemporary boxplot are right-skew and more spread out meaning that there is a
high frequency of higher votes for mascot being an extrovert. Meanwhile, samples for Sophisticated and Unpretentious
categories are more condensed in negative and positive direction respectively, which shows us that while listening to
songs from these categories participants were more consistent on attributing the extraversion traits to the mascot.
\par \textbf{Agreeableness.} There is a significantly large effect of Sophisticated and Contemporary, and Contemporary
and Unpretentious groups on measuring the mascot’s agreeableness facets with p<00.1. According to Panel A in Figure A.3,
it discriminates most of the Contemporary samples from samples of the other two categories having 75\% of votes in inaccurate direction.
\par \textbf{Conscientiousness.} For Mascot’s Conscientiousness trait, the results revealed the effect of the same
groups as we have observed for an Agreeableness personality trait. Scores for Mascot having Conscientiousness personality
traits increased sharply during listening Sophisticated (median = 3.2) and Unpretentious (median = 3.2) music in
comparison to scores for Contemporary music (median = 2.8) (for more details see Appendix A.3). The Wilcoxon test using
Bonferroni correction confirms the statistically significant difference between Sophisticated and Contemporary
(Z = 3.0545, p<0.01, p<0.01), and Unpretentious and Contemporary (Z = -2.7863, p<0.01, padj<0.05) categories.
\par \textbf{Neuroticism.} Panel N of Figure A.3 shows that the scores given for Contemporary music while assessing
Neuroticism personality traits are highest with the average scores (median = 3.1) higher than 73\% of scores given for
the other two categories. According to the Wilcoxon test with Bonferroni adjustment, there are two significant differences
between Sophisticated and Contemporary, and Contemporary and Unpretentious music with  Z = -3.9739, p<0.01, padj<0.01
and Z = 3.7269, p<0.01, padj<0.01 respectively (see Table~\ref{table:wilcoxMS1}). The correlation between Sophisticated
and Unpretentious is not consistent enough in order to draw a conclusion. Thus, the analysis shows a substantial positive
relationship between Contemporary music and the Neuroticism personality trait of a Mascot.
\par \textbf{Openness.} Panel O of Figure A.3 shows good separation of Sophisticated with median = 3.8 and max = 4.9
from other music categories (see Appendix A.3 and Figure A.3). The samples for the mascot with a current personality
trait are well behaved. There is a large effect size between Sophisticated and Contemporary (r=0.859), and Sophisticated
and Unpretentious (r=0.630) which is also confirmed with the Wilcoxon test using Bonferroni correction with p<0.01
(see Table~\ref{table:wilcoxMS1}). However, there is also a correlation between Contemporary and Unpretentious
(Z=-3.9709, p<0.01, padj<0.01).


\subsection{Analysis of within music category study}
\label{Study2(M-S)}
Our second study for mascot-speakers interaction shows how three music categories convey each personality trait
differently. In this study, we analyze the effect of each music category separately, particularly how each personality
trait is assessed differently within one music condition. Based on Friedman’s tests there is a very significant
difference between five personality traits within each music category with p< 0.01. Sophisticated condition has the
most distinctive measurements of mascot’s personality trait (Z = 66.573, df = 4, p< 0.01).

\begin{longtable}{ |p{2cm}| p{1cm}|p{0.5cm}|p{1.7cm}| }
\captionsetup{width=13.5cm}
\caption{The results from Friedman test for all Five Personality traits in case of Mascot-Speakers interaction }
\label{table:friedmanMS2} \\
\hline
 \multicolumn{1}{| c}{\textbf{Personality trait }} 
  & \multicolumn{1}{| c}{\textbf{$\chi^2$}}  
  & \multicolumn{1}{| c}{\textbf{df}} 
  & \multicolumn{1}{| c |}{\textbf{p}}  \\
\hline 
\endfirsthead
\multicolumn{4}{c}%
{\tablename\ \thetable\ -- \textit{Continued from previous page}} \\
\hline
 \multicolumn{1}{| c}{\textbf{Personality trait }} 
  & \multicolumn{1}{| c}{\textbf{$\chi^2$}}  
  & \multicolumn{1}{| c}{\textbf{df}} 
  & \multicolumn{1}{| c |}{\textbf{p}}  \\
\hline
\endhead
\hline \multicolumn{4}{r}{\textit{Continued on next page}} \\
\endfoot
\hline
\endlastfoot
Sophisticated		&66.573	&4	&p<0.01 \\
Contemporary		&44.395	&4	&p<0.01\\
Unpretentious		&57.433	&4	&p<0.01 \\
 \hline 
\end{longtable}

\par \textbf{Sophisticated.} On average, Sophisticated music conveyed Openness personality trait with padj < 0.01
and large effect sizes reported on Table~\ref{table:wilcoxMS2}. Moreover, openness, agreeableness, and neuroticism
personality traits shows significant difference between each other and all other personality traits in the group.
For neuroticism personality trait, there is a negative correlation to sophisticated music with p < 0.01.
\par \textbf{Contemporary.} According to the Figure A.4, there is a clear separation of extraversion samples from all
other personality traits. Table~\ref{table:wilcoxMS2}shows a sufficiently great difference between extraversion and all
other personality traits for Contemporary music with adjusted p< 0.01.
\par \textbf{Unpretentious.} There is a negative relationship between neuroticism personality trait and Unpretentious
category with p < 0.01 (see Table~\ref{table:wilcoxMS2}). The median values of all other personality traits slightly
differ from each other, resulting in insignificant difference between them (see Appendix A.4).

\section{Analysis of Mascot-Mascot interaction}
\label{M-M}
This section covers the analysis of the effect of each level of vibration on the perception of the personality trait
of approaching mascot which triggers this vibration. In subsection 6.3.1, we examine the results of statistical tests
for within each personality trait and in subsection 6.3.2 the analysis of within each vibration level. In addition,
from now on each vibration level is abbreviated accordingly. For example, the vibration with 500-millisecond duration
is abbreviated as ‘Level-5’, and with 100-millisecond long as ‘Level-1’ and so on.

\subsection{Analysis of within personality trait study}
\label{Study1(M-M)}
The first study covers the analysis of each personality trait and the variation of all vibration levels within this
personality group. In this study, we analyze each personality trait separately and see which vibration level conveys
it most. According to Friedman’s tests, all vibration levels affect the measurements of the mascots’ personality traits
with a significance p < 0.001 (see Table~\ref{table:friedmanMM1}) except openness personality which has 0.05 level of
significance. As a result, the different levels of vibration can be associated with the personality trait that mascot
presents. The highest difference of measurements between vibration levels was observed for Conscientiousness and
Extraversion personality traits with $\chi^2$ = 43,236, df = 4, p< 0.01 and $\chi^2$ = 30,82, df = 4, p< 0.01 respectively.
Further, the groups where this difference is concentrated will be reported for each personality trait.

\begin{longtable}{ |p{3cm}| p{1cm}|p{0.5cm}|p{1.7cm}| }
\captionsetup{width=13.5cm}
\caption{The results from Friedman test for all Five Personality traits in case of Mascot-Mascot interaction }
\label{table:friedmanMM1} \\
\hline
 \multicolumn{1}{| c}{\textbf{Personality trait }} 
  & \multicolumn{1}{| c}{\textbf{$\chi^2$}}  
  & \multicolumn{1}{| c}{\textbf{df}} 
  & \multicolumn{1}{| c |}{\textbf{p}}  \\
\hline 
\endfirsthead
\multicolumn{4}{c}%
{\tablename\ \thetable\ -- \textit{Continued from previous page}} \\
\hline
 \multicolumn{1}{| c}{\textbf{Personality trait }} 
  & \multicolumn{1}{| c}{\textbf{$\chi^2$}}  
  & \multicolumn{1}{| c}{\textbf{df}} 
  & \multicolumn{1}{| c |}{\textbf{p}}  \\
\hline
\endhead
\hline \multicolumn{4}{r}{\textit{Continued on next page}} \\
\endfoot
\hline
\endlastfoot
Extraversion		&30.82	&4	&p<0.01 \\
Agreeableness		&19.767	&4	&p<0.01 \\
Conscientiousness	& 43.236	&4	&p<0.01 \\
Neuroticism		&28.212	&4	&p<0.01\\
Openness			&11.169	&4	&p<0.05 \\
 \hline 
\end{longtable}

\par \textbf{Extraversion.} According to Table~\ref{table:wilcoxMM1}, in comparison to all other levels, Level5 and
Level-4 show a significant correlation to the attributed mascot’s extraversion personality. Based on the analysis,
there is no relationship between Level-5 and Level-4. Both of these levels were highly measured as the behavior of an
extravert mascot. While the 50\% of votes for Level-5 are concentrated between “accurate” and “very accurate”
scales (M = 4.0), the lowest vibration levels (M = 2.0, M = 2.0) showed the lower perception of a mascot as an extravert
(see Appendix A.5).
\par \textbf{Agreeableness.} Both Level-2 and Level-3 revealed a significant positive correlation to mascot being
assessed as agreeable with padj < 0.05 and large effect sizes (see Table~\ref{table:wilcoxMM1}). According to
Figure A.5, half of the samples for Level-2 are located around a “very accurate” scale. Overall, the majority of votes
given for Level-2 and Level-3 attributing agreeableness personality, are higher than most votes given for other
vibration levels (see Figure A.5).
\par \textbf{Conscientiousness.} The analysis revealed strong positive relationships between Level-4 and the levels
Level-3 and Level-4 with padj < 0.01 (see Table~\ref{table:wilcoxMM1}). The median values for Level-3 (M = 3.8) and
Level-4 (M = 4.0) are high enough to conclude that there is a strong effect of these levels to perceive mascot as Conscientious.
\par \textbf{Neuroticism.} For a Neuroticism personality trait, there is a good separation of samples for Level-1
from all other vibration levels (see Figure A.5). According to Table~\ref{table:wilcoxMM1}, Level-1 is correlated
with all other levels. Despite the significant relationship between Level-1 and Level-3, the outlies from Level-3 may
affect the potential overlay dispersion for larger sample sizes.
\par \textbf{Openness.} The Wilcoxon test revealed a significant relationship between levels one and two, and the
openness personality trait with p < 0.05. The median values for Level-5, Level4, Level-2, and Level-1 imply the overall
similarity of votes concentrated on the “neutral” scale (see Figure A.5). Moreover, the outlier for Level-3 vibration
may change the overall range of samples to more negative direction (i.e lower scales).

\subsection{Analysis of within vibration level study}
\label{Study2(M-M)}
In our first study, we analyzed the variety and the difference of each vibration level within specific personality
traits individually. In this subsection, we report the results from a different perspective. The second study analyzes
each vibration level individually and the difference of each personality trait within a specific vibration level.
According to the Friedman test, there is a statistically significant difference between the measurements of each
personality trait within each vibration level. In addition, Level-4 and Level-3 revealed that the measurements of
personality traits are very different from each other within these two vibrations ($\chi^2$ = 46.603, p< 0.01 and
$\chi^2$ =35.165, p< 0.01 respectively)

\begin{longtable}{ |p{2cm}| p{1cm}|p{0.5cm}|p{1.7cm}| }
\captionsetup{width=13.5cm}
\caption{The results from Friedman test for all Five Personality traits in case of Mascot-Mascot interaction}
\label{table:friedmanMM2} \\
\hline
 \multicolumn{1}{| c}{\textbf{Personality trait }} 
  & \multicolumn{1}{| c}{\textbf{$\chi^2$}}  
  & \multicolumn{1}{| c}{\textbf{df}} 
  & \multicolumn{1}{| c |}{\textbf{p}}  \\
\hline 
\endfirsthead
\multicolumn{4}{c}%
{\tablename\ \thetable\ -- \textit{Continued from previous page}} \\
\hline
 \multicolumn{1}{| c}{\textbf{Personality trait }} 
  & \multicolumn{1}{| c}{\textbf{$\chi^2$}}  
  & \multicolumn{1}{| c}{\textbf{df}} 
  & \multicolumn{1}{| c |}{\textbf{p}}  \\
\hline
\endhead
\hline \multicolumn{4}{r}{\textit{Continued on next page}} \\
\endfoot
\hline
\endlastfoot
Level-1		&24.008	&4	&p<0.01 \\
Level-2		&24.525	&4	&p<0.01 \\
Level-3		&35.165	&4	&p<0.01 \\
Level-4		&46.603	&4	&p<0.01 \\
Level-5		&24.1	&4	&p<0.01 \\
 \hline 
\end{longtable}

\par \textbf{Level-1.} The Wilcoxon Signed-rank test showed a positive relationship between Level-1 and a
neuroticism personality trait with a p<0.05 and large effect sizes (see Table~\ref{table:wilcoxMM2}). The median
rates given for extraversion, agreeableness, conscientiousness and openness personality traits are relatively similar,
namely 2.0, 2.5, 2.5, 2.3 (see Appendix A.6).
\par \textbf{Level-2.} Moreover, there is also a positive correlation between Level-2 and an agreeableness personality
trait with a p< 0.05. The samples from agreeableness are separated with 60\% of votes still being higher than all
votes given for all other personality traits (see Appendix A.6)
\par \textbf{Level-3.} Based on the statistical tests, Level-3 conveys two personality traits: agreeableness and
conscientiousness with p< 0.05. Moreover, there are outliers reported for neuroticism and openness personality traits
that can change the dispersion for larger sample sizes (see Figure A.6).
\par \textbf{Level-4.} Level-4 is also associated with two personality traits such as conscientiousness and extraversion
with a very significant p < 0.01.
\par \textbf{Level-5.} There is a positive relationship between the vibration with the longest duration and extraversion
personality trait with a p< 0.05. The median values for all other personality traits are at about the same scale
lower than neutral votes (see Appendix A.6).

\section{Analysis of Mascot-Tablet interaction}
\label{M-T}
The section describes the impact of the screen color of a tablet on the assessment of the personality trait that
mascot was assigned. Subsection 6.4.1 shows the analysis of within personality study and subsection 6.4.2 of condition study

\subsection{Analysis of within personality trait study}
\label{Study1(M-T)}
In our first study, we focus on each personality trait separately. Friedman’s tests reveal a significant (p < 0.01)
difference in the measurements of the mascot’s personality depending on which background color will be displayed
on the tablet (see Table~\ref{table:friedmanMT1}). The most significant effect if the predefined screen colors are
observed in agreeableness and openness personality traits with $\chi^2$ = 52.895, df = 4, p< 0.01 and $\chi^2$ = 37.725,
df = 4, p< 0.01 respectively.

\begin{longtable}{ |p{2.7cm}| p{1cm}|p{0.5cm}|p{1.7cm}| }
\captionsetup{width=13.5cm}
\caption{The results from Friedman test for all Five Personality traits in case of Mascot-Tablet interaction}
\label{table:friedmanMT1} \\
\hline
 \multicolumn{1}{| c}{\textbf{Personality trait }} 
  & \multicolumn{1}{| c}{\textbf{$\chi^2$}}  
  & \multicolumn{1}{| c}{\textbf{df}} 
  & \multicolumn{1}{| c |}{\textbf{p}}  \\
\hline 
\endfirsthead
\multicolumn{4}{c}%
{\tablename\ \thetable\ -- \textit{Continued from previous page}} \\
\hline
 \multicolumn{1}{| c}{\textbf{Personality trait }} 
  & \multicolumn{1}{| c}{\textbf{$\chi^2$}}  
  & \multicolumn{1}{| c}{\textbf{df}} 
  & \multicolumn{1}{| c |}{\textbf{p}}  \\
\hline
\endhead
\hline \multicolumn{4}{r}{\textit{Continued on next page}} \\
\endfoot
\hline
\endlastfoot
Extraversion		&28.841	&4	&p<0.01 \\
Agreeableness		&52.895	&4	&p<0.01 \\
Conscientiousness	&32.891	&4	&p<0.01 \\
Neuroticism		&30.466	&4	&p<0.01 \\
Openness			&37.725	&4	&p<0.01 \\
 \hline 
\end{longtable}

\par \textbf{Extraversion.} According to the Wilcoxon tests using Bonferroni correction, for extraversion personality,
the most significant difference was observed in orange and the colors blood-red and pink with padj < 0.05
(see Table~\ref{table:wilcoxMT1}). Figure A.7 shows that samples from yellow and orange colors are separated
from blood-red and pink samples. Moreover, based on median value, 50\% of samples for orange are concentrated around
an “accurate” score (M = 4.2, max = 5.0) (see Appendix A.7).
\par \textbf{Agreeableness.} Table~\ref{table:wilcoxMT1} shows a good separation of turquoise and pink colors
from all others implying that there is a positive correlation to the agreeableness personality traits having
padj < 0.01. However, there is no strong difference between turquoise and pink colors. Two potential outliers for
orange and turquoise colors indicate that for larger sample sizes the dispersion of turquoise may spread in a negative
and for orange in a positive direction.
\par \textbf{Conscientiousness.} According to Table~\ref{table:wilcoxMT1}, there is a significant relationship between
tablet displaying turquoise color and mascot being assessed as conscientiousness with padj < 0.05. The median values
for yellow (M = 2.7), pink (M = 2.8) and orange M = 2.7) colors are concentrated around the “neutral” scale. However,
there are three large potential outliers that would have an effect on the overall measurements of a mascot's
personality for larger sample sizes.
\par \textbf{Neuroticism.} According to Figure A.7, there is a great separation of blood-red samples from all other
screen colors. Moreover, blood-red is statistically significantly different from other colors with a p < 0.01. There
are some extreme values observed for yellow and turquoise colors (see Figure A.7). All colors except blood-red are
evaluated as a behavior of mascots being low in neuroticism personality with median values around 2.5 (see Appendix A.7).
\par \textbf{Openness.} According to the Wilcoxon test using Bonferroni correction, during the measurements of a
mascot's 'openness to experience' personality trait, the yellow is significantly different from all other screen
colors with padj < 0.05 (see Table~\ref{table:wilcoxMT1}). Most of the samples of yellow color are discriminated
from all others concentrating around an “accurate” scale with a median = 4.0 (see Appendix A.7 and Figure A.7).

\subsection{Analysis of within screen color study}
\label{Study2(M-T)}
For the second study, we focus on each color separately by considering the different measurements of five
personality traits within each screen color. Friedman’s test shows that Turquoise color has the most significant
effect on all personality traits (Z=46.199, df=4, p< 0.01). Orange color has the least substantial effect on
overall measurements of all five personality traits (Z=19.992, df=4, p< 0.01).

\begin{longtable}{ |p{3cm}| p{1.7cm}|p{0.5cm}|p{1.7cm}| }
\captionsetup{width=13.5cm}
\caption{The results from Friedman test for all Five Personality traits in case of Mascot-Tablet interaction}
\label{table:friedmanMT2} \\
\hline
 \multicolumn{1}{| c}{\textbf{Personality trait }} 
  & \multicolumn{1}{| c}{\textbf{$\chi^2$}}  
  & \multicolumn{1}{| c}{\textbf{df}} 
  & \multicolumn{1}{| c |}{\textbf{p}}  \\
\hline 
\endfirsthead
\multicolumn{4}{c}%
{\tablename\ \thetable\ -- \textit{Continued from previous page}} \\
\hline
 \multicolumn{1}{| c}{\textbf{Personality trait }} 
  & \multicolumn{1}{| c}{\textbf{$\chi^2$}}  
  & \multicolumn{1}{| c}{\textbf{df}} 
  & \multicolumn{1}{| c |}{\textbf{p}}  \\
\hline
\endhead
\hline \multicolumn{4}{r}{\textit{Continued on next page}} \\
\endfoot
\hline
\endlastfoot
Yellow		&38.142	&4	&p<0.01 \\
Orange		&19.992	&4	&p<0.01 \\
Turquoise		&46.199	&4	&p<0.01 \\
Blood-Red	&29.95	&4	&p<0.01 \\
Pink			&21.68	&4	&p<0.01 \\
 \hline 
\end{longtable}

\par \textbf{Yellow.} The Wilcoxon Sign rank test with applied Bonferroni correction revealed that there us a
positive relationship between yellow screen color and openness personality trait with p < 0.01. The possible outliers
for neuroticism may change the dispersion of samples for a larger number of participants. However, since the outliers
are not very far from higher adjacent values, we can assume that it will not affect the correlation between openness
and yellow color.
\par \textbf{Orange.} As observed in Friedman’s test, there are still statistical significant results for orange
screen colors. However, in comparison to all other screen colors, the variation of measurements of personality traits
is not substantially different from each other. The Wilcoxon test shows the only three distinguishable groups
between extraversion, agreeableness, and neuroticism personality traits within the orange color condition
(see Table~\ref{table:wilcoxMT2}). However, there is no separation of one personality from others so that we can
correlate it with orange screen color. The extraversion seems to have the highest overall positive opinions
with median = 4.2 (see Appendix A.8)
\par \textbf{Turquoise.} Turquoise screen color shows the most amount of distinguishable effects of colors
on all personality traits. Particularly, conscientiousness and agreeableness personality traits are the most
conveyed by turquoise colors with a p< 0.05 (see Table~\ref{table:wilcoxMT2}). Moreover, the median value for
conscientiousness (M = 4.2) is higher than for agreeableness (M = 3.7) and all other personality traits
(see Appendix A.8). Two outliers are reported in Figure A.8. We assume that for neuroticism, it will not affect the
overall result since the samples are already located toward negative opinions. For agreeableness,
the overall dispersion may affect the result which, therefore, may make the turquoise color even more influential
on conveying a conscientiousness personality trait.
\par \textbf{Blood-red.} For blood-red color, there is a strong positive relationship between blood-red and neuroticism
personality trait with p < 0.05. The effect sizes and visually presented boxplots show a great separation of
neuroticism samples from all other personality traits (see Table~\ref{table:wilcoxMT2}and Figure A.8).
In fact, only one outlier reported for an openness personality trait.
\par \textbf{Pink.} There is a positive relationship between pink and agreeable personality traits with p < 0.05.
However, we could not find a statistical difference between pink color and such personality traits ads agreeableness
and conscientiousness. For pink color, there are so many outliers for almost every personality trait except
for agreeableness. This can affect the overall spread of samples for larger sample sizes.
