\chapter{Introduction}
\label{ch:introduction}

\section{Motivation}
\label{sec:motivation}
The rapid development of the area of Information Technology led to the
emergence of a new paradigm known as "Internet of Things".
IoT adds a new dimension by focusing not only on the interaction between
humans and devices but also on the devices themselves.
However, the concept of IoT does not ensure the effective discovery of the
objects and the better reaction to the states of other objects.
A new paradigm "Social Internet of Things" (SIoT) introduces more autonomous
interaction between "things" by applying the notion of Social Relationship of
things rather than explicit users' instructions.
However, the ambition of having autonomous interaction of things increases
the complexity of a system without having users in mind.
Throughout this paper, "things" based on the SIoT concept will be referred as
"social things" or "social devices" for convenience.

Social things interacting with each other act without making visible to the
user the decisions made by a system.
Thus, it is unrealistic to assume that the autonomous decisions made by
social things will always be in line with users' expectations.
Given this problem, we are motivated to find and investigate an approach
that will take the user's context into account.
Tus, the system that achieves the cooperation of social things
where things are assigned with unique personalities is proposed.
A user being able to configure the personality of the social device will be ensured about
the consistent behavior of devices where the services provided by these devices will remain dynamic.
We assume that the concept of the personality with predefined actions increases a user's
awareness of a system and serves as an interaction mechanism inside the SIoT environment.

The system implemented in this thesis was inspired by the "Autonomous Cooperation of Social Things"
paper~\cite{okada2016autonomous},
where authors achieve the interaction between mascot and bench as case
studies by representing to cooperation among private and public things.
A more detailed description of a system and its contribution is described
in Section~\ref{sec:Autonomous interaction of things}.
Additionally, authors introduce social things concept where things have unique personalities.
However, they do not use personality from a design perspective.
In their system, a mascot interacts with a bench which shows only static behavior based on the user's position in space.
We plan to apply the personality concept as a design process and bring it to the
behavior of social things where they will react to each other dynamically based on the personality of a social device.
The use of the personality concept as a design process may help users to
create a mental model about the behavior of social devices and its environment.
Thus, we envision that the concept of personality with preset actions will influence
not only the interaction between social things but also between the user and social things.

\section{Research Goals}
\label{sec:research-goals}
As we mentioned in the previous Section~\ref{sec:motivation}, from the user perspective, social devices introduced
in the paper~\cite{okada2016autonomous} have static behavior.
Users are not able to configure devices or input some information and see the output accordingly.
Considering the role of the human in the SIoT environment, in our system, the personality with preset
actions serves as an interface for interaction between user and its environment including social devices.

Therefore, as an input, we use the configuration of mascot's personality and proximity
information of the user's movements.
As an output, we use actions that all social things in the environment display
(i.e change of light, vibration, music play, and screen color alteration) based
on the personality information that the user configured.

To wrap it up, we extended the system introduced in a paper~\cite{okada2016autonomous} by adding:
\begin{itemize}
    \item More case studies such as mascot-lamp, mascot-speakers, mascot-mascot, and mascot-tablet interactions.
    Thus, the cooperation among private and public things proposed by previous system~\cite{okada2016autonomous}
    is now extended to more interaction types (see Section~\ref{sec:Identifying case studies and actions})
    \item Preset actions for each case-study:
    \begin{itemize}
        \item Yellow, orange, turquoise, pink, blood-red lighting color for a mascot-lamp interaction.
        \item Sophisticated, contemporary, unpretentious music for mascot-speakers interaction.
        \item Vibrations with five different durations starting from 100 to
        500 milliseconds per time for mascot-mascot interaction.
        \item Yellow, orange, turquoise, pink, blood-red screen background for mascot-tablet interaction.
    \end{itemize}
\end{itemize}
Chapter~\ref{ch:concept} is completely dedicated to the design of a system and to
the concepts applied in this research in more detail.

Given the extended SIoT system, the main focus lies on how people interpret
the interaction between user and social devices.
A person holding a mascot changes the state of all devices in the system by his movements.
Thus, the question that arises is: "How people will interpret the following actions of things such as
lighting color change, music play, vibration duration and screen color
alternation in the context of personality traits?"
We suppose that the interactions between a mascot and other social
devices give a descriptive clue about the personality of this mascot.
Therefore, we empirically investigate each action individually and analyse the impact of predefined
actions on the measurements of mascot's personality and the personality trait
that these actions convey (see Chapters~\ref{ch:results}, ~\ref{ch:discussions} and ~\ref{ch:conclusion}).
Thus, for each case-study, the goal is to settle the question of how people understand devices' actions
and measure the personality trait of the mascot based on the interactions between these social devices.

\section{Outline of Thesis}
\label{sec:outline-of-thesis}
After presenting motivation and research goals (see Chapter~\ref{ch:introduction}), in Chapter~\ref{ch:related-work},
an overview of scientific publications that are relevant to the thesis is given.
The related work chapter also describes current works that inspired us to investigate further
and methodologies are applied to expand the existing system.

Chapter~\ref{ch:concept} introduces the papers that aid us to come up with case-studies;
identify personality traits that each mascot will be assigned;
identify such actions as vibration, music play, lighting,
and screen color change based on the personality trait concept.

Chapter~\ref{ch:implementation} describes the implementation details of the system providing the
interaction between social things with preset personality.
The implementation chapter includes client and server architectures,
the overall workload of a system, software, and hardware used to implement the system.

Chapter~\ref{ch:user-study} includes the user study, namely, describes the design of
experiments, procedures, tasks, and materials used to conduct the experiment.

Chapter~\ref{ch:results} presents a statistical analysis of all four case-studies.
The analysis measured from two perspectives framed into two separate sub-studies.

Chapter~\ref{ch:discussions} covers the discussion of statistical results for each case-study.

The final Chapter~\ref{ch:conclusion} gives an overview of the results, contributions,
possible limitations of a study, and future works that can extend existing study.