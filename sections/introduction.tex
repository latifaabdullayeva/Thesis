\chapter{Introduction}
\label{ch:introduction}
This Master’s thesis presents the ... toolkit .
The toolkit allows to ...
For this purpose, methods ... will be presented.

\section{Motivation}
\label{sec:motivation}
The rapid development of the area of Information Technology led to the
emergence of a new paradigm known as "Internet of Things".
IoT adds a new dimension by focusing not only on the interaction between
humans and devices but also between devices themselves.
However, the concept of IoT does not ensure the effective discovery of
objects and better reaction to the states of other objects.
Unfortunately, in recent IoT products, users play a key role for interaction between objects.
A new paradigm "Social Internet of Things" (SIoT) introduces more autonomous interaction between "things"
by applying the notion of Social Relationship of things rather than explicit users' instructions.
Throughout this paper "things based on SIoT concept will be called "social things" or "social devices" for convenience.
However, the ambition of having autonomous interaction of things increases
the complexity of a system without having user in mind.

Social things interacting with each other act without making visible to the user the decisions made by a system.
Thus, it is unrealistic to assume that the autonomous decisions made by social things
will always be in line with users' expectations.
Given this problem, we are motivated to find and investigate on the approach that will take user's context into account.
We propose the system that achieves the cooperation of social things where
things are assigned with unique personalities.
A user being able to configure personality of social device, will be ensured about
consistent behavior of devices, moreover, the services provided by these devices will remain dynamic.
We assume that personality with preset case-studies will increase user's
awareness of a system and serve as an interface inside SIoT environment.

Our prototype is inspired by the "Autonomous Cooperation of Social Things" paper ~\cite{okada2016autonomous},
where authors achieve the interaction between mascot and as case studies
by representing to cooperation among private and public things.
We describe this paper and its contribution in our research in
section ~\ref{sec:Autonomous interaction of things} in more details.
Moreover, authors introduce social things concept where things have unique personalities.
However, they do not use personality from design perspective.
In their prototype mascot having three color conditions interact with a bench
which shows only static behavior on user's position in a space.
We are planning to apply personality as a design process, and bring it to the behavior of social
thing where they will react to each other dynamically based on the personality of social device.
Thus, we envision that the concept of personality with preset actions will influence
not only the interaction between social thing but also between user and social environment.

\section{Research Goals}
\label{sec:research-goals}
As we mentioned in previous section [~\ref{sec:motivation}], in prototype introduced by
paper~\cite{okada2016autonomous}, social devices have static behavior.
To be more precise, from autonomous cooperation perspective they are dynamic, i.e \ it
change dynamically their state based on information transferred via inter-thing communication.
However, these devices become static from user perspective.
Users are not able to configure devices or input some information and see the output accordingly.
Considering the role of the human in the SIoT, in our prototype, personality with preset actions
serve as an interface for interaction between user and its environment including social devices.
Therefore, sd sn input, we use the configuration of Mascot's personality and proximity information of user's movements.
As an output, we use actions that all social things in the environment displays (i.e change of light,
vibration, music play and screen color change) based on the personality information that user configured.

To wrap it up, we extended the system introduced in paper~\cite{okada2016autonomous} by adding:
\begin{itemize}
    \item more case studies: Mascot-Lamp, Mascot-Speakers, Mascot-Mascot and Mascot-Tablet interactions
    \item preset actions for each case-study:
    \begin{itemize}
        \item Yellow, orange, turquoise, pink, blood-red lighting color for Mascot-Lamp interaction
        \item Sophisticated, contemporary, unpretentious music for Mascot-Speakers interaction
        \item Vibrations with five different durations starting from 100 to
        500 milliseconds per time for Mascot-Mascot interaction
        \item Yellow, orange, turquoise, pink, blood-red background screen for Mascot-Tablet interaction
    \end{itemize}
\end{itemize}
Chapter~\ref{ch:concept} is completely dedicated to the design of our prototype and
describes all concepts applied in our research in more details.

The main focus lies on how people interpret the interaction between user and social devices.
A person holding a Mascot changes the state of all devices in our system by his movement.
Thus the question that interests us is "How people will interpret the following actions of things such as
lighting color, music, vibration duration and screen color in the context of personality traits?"
We suppose that the interactions between Mascot and other social
devices give descriptive clues about personality of that Mascot.
Therefore, in our user study, we investigate each action individually and see which personality trait it conveys most.
Thus, for each case-study, the goal is to settle the question of how people understand devices' actions
and measure personality trait of the Mascot based on the interactions between these social things.

\section{Outline of Thesis}
\label{sec:outline-of-thesis}
The thesis is structured in the following way.
After presenting motivation and research goals (see chapter~\ref{ch:introduction}).
In chapter~\ref{ch:related-work}, we give an overview of scientific publications that are relevant to our thesis.
Related work chapter also describes current works that inspired us to investigate further
and methodologies that we applied to expand existing system.
Chapter~\ref{ch:concept} introduces papers that aid us to come up with case-studies;
identify personality traits that each Mascot will be assigned;
identify of such actions as vibration, music play, lighting change
and screen color change based on the personality trait concept.
Chapter~\ref{ch:implementation} describes the implementation details of prototype providing the
interaction between social things and with preset personality.
Implementation chapter includes frontend and backend architectures,
overall workload of a system, software and hardware used to implement the prototype.
Chapter~\ref{ch:user-study} includes the user study, namely, describes the design of
experiments, procedures, tasks and materials used to conduct experiment.
Chapter~\ref{ch:results}, presents statistical analysis of all four case-studies.
The analysis measured from two perspectives framed into two separate studies.
Chapter~\ref{ch:discussions} covers the discussion of statistical results for each case-study.
In the final chapter~\ref{ch:conclusion}, we give an overview of our results, contributions,
possible limitations and future works that can extend our study.