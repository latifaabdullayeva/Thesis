\tikzstyle{block} = [rectangle, draw, fill=blue!20, 
    text width=30em, text centered, rounded corners, minimum height=4em]
\tikzstyle{line} = [draw, -latex']
\tikzstyle{cloud} = [draw, ellipse,fill=red!20, node distance=3cm,
    minimum height=2em]

\chapter{Implementation}

In this study, we implemented a prototype that achieves the autonomous interaction of social things by providing them with user-predefined parameters such as the personality of a Mascot. Further, given these parameters, Mascot starts to influence the state of other devices.

The chapter is structured in the following way: In section ------, we give an overview of the prototype from the user perspective which includes two subsections: one explains steps the user takes and how it affects the state of other devices, and the other describes the limitations that users have when using the prototype. Section ------ introduces the architecture of the prototype, namely, the workflow of the devices-server and server-APIs interactions. Section  ------ describes the software and hardware that were used to implement the prototype, more specifically, it gives an overview of each implemented application, the list of used frameworks and APIs and the hardware requirements that the prototype needs to meet.

\section{Overview of the prototype}

\subsection{User-side description of a system}
When the user runs the application for the first time, it displays the form that the user has to fill in order to configure his Mascot. For that, the user needs to follow the steps: he can choose the beacon ID that will help this application to measure the distance between devices; he can give his Mascot a custom name; he is required to choose one out of five personalities displayed on the screen.

\subsection{User restrictions}

\begin{figure}
\centering
\begin{tikzpicture}[node distance = 2cm, auto]
    % Place nodes
    \node [block] (init) {When the app launches for the first time, it searches for a services/server using NSD API};
    \node [block, below of=init] (identify) {Looks for all beacon tags that are located in the room using AltBeacon Library};
    \node [block, below of=identify] (evaluate) {Initialises the Mascot and defines the beacon tag, the name and the personality of its Mascot};
    \node [block, below of=evaluate] (evaluate2) {Sends request to the server in order to save all data about itself in the database};
    \node [block, below of=evaluate2] (evaluate3) {Measures distance from itself to all other devices in the system using AltBeacon Library};
    \node [block, below of=evaluate3] (evaluate4) {Sends all distance data to the server to save them in database};
    \node [block, below of=evaluate4] (evaluate5) {Gets from a server the informations about whether both endpoints are Mascots and the distance between them less or equal to  50 cm};
    \node [block, below of=evaluate5] (evaluate6) {Get from a server the personality of an approaching Mascot};
    \node [block, below of=evaluate6] (stop) {Vibrates the phone based on the personality vibration level of approaching Mascot};
    % Draw edges
    \path [line] (init) -- (identify);
    \path [line] (identify) -- (evaluate);
    \path [line] (evaluate) -- (evaluate2);
    \path [line] (evaluate2) -- (evaluate3);
    \path [line] (evaluate3) -- (evaluate4);
    \path [line] (evaluate4) -- (evaluate5); 
    \path [line] (evaluate5) -- (evaluate6);
    \path [line] (evaluate6) -- (stop);
  
\end{tikzpicture}
\caption{The interaction between Android Mascot application with a server with the libraries and APIs that it is using}
\end{figure}

	
\begin{table}
\centering
\begin{tabular}{ | m{8em} | m{6em}| m{17em} | } 
\hline
 Column name & Column type & Description \\ 
\hline 
device id & d & unique auto-generated value \\
\hline 
beacon uuid & d & the ID of beacon that user assigned to its device \\
\hline 
device type & d & which can be Mascot, Lamp, or Tablet) \\
\hline 
device name & d & a custom given name for only Mascots \\
\hline 
device personality & d & the personality that the user decided to attach for Mascot. This column is a foreign key of the “personality id” column in “Personality” table \\
\hline
\end{tabular}
\caption{Devices Table}
\end{table}
















   