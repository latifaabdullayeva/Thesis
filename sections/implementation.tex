\tikzstyle{block} = [rectangle, draw, fill=blue!20, 
    text width=30em, text centered, rounded corners, minimum height=4em]
\tikzstyle{line} = [draw, -latex']
\tikzstyle{cloud} = [draw, ellipse,fill=red!20, node distance=3cm,
    minimum height=2em]

\chapter{Implementation}
\label{ch:implementation}
This chapter is structured in the following way.
In section~\ref{sec:prototype-presentation} we will take a look at the project structure and
how all components communicate with each other.
Section~\ref{sec:configuration} describes how the project can be deployed and configured.
Section~\ref{sec:system-architecture} covers system architecture, namely the
communication protocols, frameworks and APIs that are used in the prototype.

\section{Prototype Presentation}
\label{sec:prototype-presentation}

\section{Configuration}
\label{sec:configuration}

\section{System Architecture}
\label{sec:system-architecture}










\begin{figure}
\centering
\begin{tikzpicture}[node distance = 2cm, auto]
    % Place nodes
    \node [block] (init) {When the app launches for the first time, it searches for a services/server using NSD API};
    \node [block, below of=init] (identify) {Looks for all beacon tags that are located in the room using AltBeacon Library};
    \node [block, below of=identify] (evaluate) {Initialises the Mascot and defines the beacon tag, the name and the personality of its Mascot};
    \node [block, below of=evaluate] (evaluate2) {Sends request to the server in order to save all data about itself in the database};
    \node [block, below of=evaluate2] (evaluate3) {Measures distance from itself to all other devices in the system using AltBeacon Library};
    \node [block, below of=evaluate3] (evaluate4) {Sends all distance data to the server to save them in database};
    \node [block, below of=evaluate4] (evaluate5) {Gets from a server the informations about whether both endpoints are Mascots and the distance between them less or equal to  50 cm};
    \node [block, below of=evaluate5] (evaluate6) {Get from a server the personality of an approaching Mascot};
    \node [block, below of=evaluate6] (stop) {Vibrates the phone based on the personality vibration level of approaching Mascot};
    % Draw edges
    \path [line] (init) -- (identify);
    \path [line] (identify) -- (evaluate);
    \path [line] (evaluate) -- (evaluate2);
    \path [line] (evaluate2) -- (evaluate3);
    \path [line] (evaluate3) -- (evaluate4);
    \path [line] (evaluate4) -- (evaluate5); 
    \path [line] (evaluate5) -- (evaluate6);
    \path [line] (evaluate6) -- (stop);
\end{tikzpicture}
\caption{The interaction between Android Mascot application with a server with the libraries and APIs that it is using}
\end{figure}