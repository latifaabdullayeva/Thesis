\chapter{Discussions (in progress)}
\label{ch:discussions}
This chapter covers discussion of the results obtained from statistical tests.
In our study, we tried to conceptualize the behavior of a mascot with four interaction types which we refer as case-studies.
We discuss the results of each use case in four sections separately.
Moreover, as we mentioned in the Chapter~\ref{ch:results}, we analyzed results from two perspectives: the study of
within personality trait and within condition.
Each study split into subsections and discussed for each case-studies separately.

\section{Mascot-lamp interaction}
\label{sec:mascot-lamp-interaction}
The study shows that the change in color of the lamp has an impact on the way
how participants interpret the personality of a mascot.
During the user study we explained participants that while interacting with the lamp,
a mascot with its behavior tries to convey specific personality trait.
Analysis shows that in this case participants did not just see the lamp as artificial light source,
but also it gives a clues about mascot's personality.
In the following subsections we will discuss which lighting colors effect participants' understanding
of mascot's personality trait.

\subsection{Discussion for the first study}
\label{subsec:discussion-for-the-first-study}

\par\textbf{Extraversion personality Trait.}
The results show that predefined lighting colors have an impact on the
way how participants measured the personality of mascots.
Almost 100\% of participants by seeing blood-red light, did not reflect any positive opinion toward the mascot
attributing Extraversion personality trait.
Participants rate yellow in comparison to turquoise, blood-red, pink lighting colors to be more Extravert.
However, the measurements of yellow in comparison to orange color do not differ substantially.

\par\textbf{Agreeableness personality Trait.}
In comparison to blood-red and orange colors, the majority of participants had a positive opinion
regarding personality of a mascot interacting with Pink, yellow, turquoise lighting to convey to cooperative,
trustworthy, modest and all other facets representing agreeableness personality trait (see Figure-facet).
They were giving the highest points to the mascot for being assessed as agreeable
in case of pink, yellow, turquoise and the lowest points when they saw orange and blood-red
lighting colors (see Figure--).

\par\textbf{Conscientiousness personality Trait.}
There was a clear separation of votes given for blood-red and orange for mascot being measured very low
on being Conscientiousness personality trait.
As with agreeableness, participants rated mascot interacting with pink, yellow, turquoise lighting to be more
Conscientiousness in comparison to blood-red and orange colors.

\par\textbf{Neuroticism personality Trait.}
By showing the participants interaction of a mascot with blood-red in comparison to any other colors,
the majority will choose blood-red as an expected behavior for Neurotic mascot.
Participants rated mascot triggering blood-red lighting to be highly anxious, angry, depressive, self-conscientious,
impulsive and vulnerable where all of these facets represent Neuroticism personality trait.

\par\textbf{Openness personality Trait.}
Mascot interacting with yellow color to be more openness only when it is compared to blood-red and pink colors.

\subsection{Discussion for the second study}
\label{subsec:discussion-for-the-second-study}

\par\textbf{Yellow lighting color condition.}
Participants seeing mascot interacting with yellow color measured personality traits as an extravert, or agreeable,
or conscientious, or open to the experience.

\par\textbf{Orange lighting color condition.}
When participants were shown orange lighting color, they measured mascots's personality as both
extravert or open to the experience.
Thus, on the one hand, the mascot conveying the orange light was perceived as friendly, sociable,
energetic, cheerful and the one which seeks excitement.
On the other hand, this color is also attributed mascot that is imaginative, artistic, adventurous,
intellectual and all other facets that represent openness personality (see Table-facet).

\par\textbf{Turquoise lighting color condition.}
Participants measured mascot interacting with turquoise lighting to be more conscientiousness and
agreeableness personality.
The facets that were rated high and represent these personality traits are trustworthiness, straightforwardness,
orderliness, dutifulness, self-disciplined and so on (see Table-facet).

\par\textbf{Blood-red lighting color condition.}
The mascot triggering the blood-red lighting is interpreted as an anxious,
highly depressive, angry, vulnerable and the one that has immoderate behavior.
Participants rated very high all above mentioned facets that constitute Neuroticism personality trait (see Table-facet).

\par\textbf{Pink lighting color condition.}
Similar to turquoise color, participants rated mascot triggering pink lighting to be
agreeableness and conscientiousness personality.
Thus, by presenting this lighting color, the mascot can be attributed as both modest, cooperative, trustworthy
which are facets that constitute agreeableness, and orderly, dutiful, discipline which belongs to conscientiousness.

\section{Mascot-speakers interaction}
\label{sec:mascot-speakers-interaction}
For mascot-speakers interaction, there are some music categories that convey the particular personality trait of a mascot.
In general, the interpretation of a mascot varies depending on which song this mascot triggered
which we will discuss in the following subsections.


\subsection{Discussion for the first study}
\label{subsec:discussion-for-the-first-study4}

\par\textbf{Extraversion personality Trait.}
Among all categories, participants measured the interaction between mascot and contemporary music
to be more Extravert.
Thus, they tended to give high scores for a mascot triggering rap and soul songs
to be more friendly, sociable, cheerful and other facets representing Extraversion personality traits (see Figure-facets).
We can assume that the noticeable rhythmic elements and a mainstream style of these genres
may lead participants to rate mascot as an Extravert.

\par\textbf{Agreeableness personality Trait.}
The contemporary music was rated very low on conveying agreeableness personality trait.
The interaction of a mascot with unpretentious and sophisticated music is high on being agreeableness.

\par\textbf{Conscientiousness personality Trait.}
The measurement of conscientiousness personality trait is similar to agreeableness.
Both sophisticated and unpretentious are measured to be more conscientiousness compared to contemporary music.
Thus, the genres from a Sophisticated category such as classical, jazz, and contemporary adult and from it such as
pop, rock \& roll, and bluegrass portrayed the mascot as being more conscientiousness.

\par\textbf{Neuroticism personality Trait.}
Among all three categories, mascot's interaction with Contemporary music is rated very high on
conveying Neuroticism personality trait such genres as rap and soul measured mascot as being more
emotionally unstable, angry, vulnerable, depressive and etc (see Table-facets).
Participants who listened to Sophisticated and Unpretentious music measured mascot relatively low on Neuroticism.

\par\textbf{Openness personality Trait.}
Overall, participants measured mascot triggering sophisticated music to be more openness.
In comparison to contemporary and unpretentious categories.
We can assume that such songs as classical, jazz contains ensembles of instruments with complex
arrangements and they have distinctive tones and performance techniques which leads to measure the
mascot as a complex, intelligent and thoughtful, the one who will enjoy non-mainstream music.

\subsection{Discussion for the second study}
\label{subsec:discussion-for-the-second-study2}

\par\textbf{Sophisticated music condition.}
The mascot that trigger speakers to play sophisticated music is measured high on conveying openness personality trait.

\par\textbf{Contemporary music condition.}
Participant measured mascot interacting with contemporary music to present Extraversion personality trait.

\par\textbf{Unpretentious music condition.}
According to participants' ratings the unpretentious music conveys agreeableness when it is compared
to all other personality traits except openness.
... Neuroticism is low
... Agreeableness (no difference) Openness

\section{Mascot-mascot interaction}
\label{sec:mascot-mascot-interaction}
The statistical analysis showed that the levels of vibration have a significant effect on the
measurement of mascot's personality.
Further, we discuss specific levels that personality traits convey.

\subsection{Discussion for the first study}
\label{subsec:discussion-for-the-first-study2}

\par\textbf{Extraversion personality Trait.}
The mascot with the longest vibration duration, namely 500 milliseconds long, was interpreted as assertive,
forceful, energetic, friendly, sociable, cheerful.
All of these facets constitute extraversion personality trait.
Moreover, Level-4 showed a second-highest, after Level-1, positive correlation to the mascot
being perceived as an extravert.
In addition, after experiments, participants commented that the longer their mascot vibrated,
the more they got an impression that approaching mascot wants to socialize with them.
This made them give higher scores for extraversion facets.
Importantly, participants were measuring all 30 facets without knowing which of them
belong to the extraversion dimension.

\par\textbf{Agreeableness personality Trait.}
Level-2 and Level-3 are more separated from all other levels showing a positive correlation
for mascot being assessed as agreeable.
This may indicate that for the majority of participants in order for mascot being assessed as modest,
cooperative, trustworthy, the vibration level that they represent should not be as assertive as we have
noticed for extraversion personality trait and not as passive as we will notice for neuroticism trait.

\par\textbf{Conscientiousness personality Trait.}
Level-3 and Level-4 reveal a positive correlation to the currently described personality trait.
Since conscientiousness personality trait is characterized by being goal-oriented (see Appendix Questionnaire),
we can assume that participants might expect from conscientious mascot to be a little bit more assertive and
forceful in order to achieve its goals.
Thus, it can explain why participants agreed that Level-4 conveys conscientious personality trait.

\par \textbf{Neuroticism personality Trait.}
The mascot was scored significantly high on being neurotic when participants experienced
the vibration with the shortest duration, namely Level-1.
This level sows a great separation from all other vibration levels.
In addition, we can assume that the longer the duration of the vibration is,
the more this mascot represents its desire to communicate with other mascots.
Communication may require additional skills such as coping with a stressful situation,
and in this kinds of situation, neurotic personality trait may be vulnerable and feel anxiety.
It gleans insight into the reason why Level-1 is highly related to a neuroticism personality trait.

\par\textbf{Openness personality Trait.}
Overall, in comparison to other traits, the results for openness personality did not show very significant differences.
The only difference were found between Level-1 and Level-2.
Therefore, the mascot with a 200-milliseconds duration of the vibration
may be perceived as an openness personality trait, only when it is compared with 100-milliseconds duration.
The conclusion that we can draw is that such personality facets as imagination, adventurousness,
intellect, liberalism hard to assess based on only vibration levels.

\subsection{Discussion for the second study}
\label{subsec:discussion-for-the-second-study3}

\par\textbf{Level-1 vibration condition.}
When the mascot approaches the participant that holds their own mascot, their mascots start to vibrate.
When their mascot vibrated with the lowest level of vibration they interpreted approaching mascot as neurotic.
Thus, Level-1 conveys anxious, depressive, vulnerable mascot.

\par\textbf{Level-2 vibration condition.}
Level 2 portrays modest, trusting, cooperative and straightforward facets which constitute
agreeableness personality traits.
This, if we want to assign agreeable personality traits to any social device the vibration
level that is a good example to convey this personality is Level-2.
We can assume that agreement could also tends to co-operate and socialize with others.
Since is a mascot comprises modest facet, in order not to disturb other mascots or their owners,
it communicate with others in the more restrained way.
Thus, we expect the levels of vibration that has neutral duration such as Level-2 and later we will see Level-3.
Because the higher is duration of vibration is, the more assertiveness it implies.
Therefore, Level-2 does not disturb other owners of a mascot in an assertive manner
which portrays the modesty and empathy traits of agreeableness.

\par\textbf{Level-3 vibration condition.}
Level-3 may convey two personality traits: agreeableness and conscientiousness.
the reason why we think Level-3 portrays agreeableness personality traits is the same as we discussed for Level-2.
regarding conscientiousness personality, it is characterized by being orderly, dutiful, goal-oriented and
therefore we would expect the tendency to be more careful with actions.
This may explain that the neutral level of vibration conveys conscientiousness personality trait.
In fact, some of these assumptions are made in order to better understand the reason behind participants' choices.
Whereas, some of the assumptions are the comments that participants made in during video watch.
Since our experiments did not include open-ended questionnaire, it can be a good idea
to take it into account during future work.

\par\textbf{Level-4 vibration condition.}
According to results Level-4 also conveys two personality dimensions: conscientiousness and extraversion.
The possible reason for Level-4 to portray conscientiousness personality trait might be the
goal-oriented, achievement-striving and competence characteristics.
On the one hand, Level-3 conveys conscientiousness personality trait because of its
underlines, dutifulness, and self-discipline characteristics which we discussed above.
On the other hand conscientiousness constitutes competence and achievements-striving facets which may be the reason
why longer vibration duration conveys conscientiousness personality.

\par\textbf{Level-5 vibration condition.}
Level 4 and level 5 they both portray extraversion personality dimension.
The assertiveness and forceful facets of a mascot may be a reason why Level-5 conveys extraversion personality.
We can assume that the mascot will force other mascot to vibrate with the longest duration until the owner react to it.
The other reason might be the mascot is active, energetic and seeks for excitement,
therefore, it tries to involve all other mascots and their owners to be social.

\section{Mascot-tablet interaction}
\label{sec:mascot-tablet-interaction}
Based on analysis, we can claim that some colors of the screen portrays specific personality traits.
In the following subsections, we discuss which colors were more excessively assessed as personality traits.
Moreover, we will give intuitive assumptions about why we think participants
associated certain colors with particular personality traits.
In fact, those assumptions are not bse on the research papers, they just help us to explain participants’
decisions in a more intuitive manner.

\subsection{Discussion for the first study}
\label{subsec:discussion-for-the-first-study3}

\par There is no strong relationship between mascot being assessed high
in \textbf{extraversion personality} and the background screen colour to the tablet displays.
yellow and orange colours can be assessed as an extrovert mascot only when they are compared to
pink and blood red colours.
however it is hard to emphasise one colour that would convey extraversion personality.

\par The analysis revealed that blood-red screen color is negatively correlated with the mascot
being measured as \textbf{agreeableness personality}.
This implies that when participants have seen the tablet with blood red screen colour measured
mascot as being very low in agreeableness.
Moreover pink and turquoise colours convey agreeable personality.
This can be explained by the agreeableness personality being kind, warm, generous,
forgiving and unselfish and being portrayed by calming colours such as pink and turquoise.
Pink colour can be associated with flowers and turquoise colour with sky or ocean for blue side
of turquoise and nature for green side.
Some participants mentioned that since these two colours remind nature, they decided to measure
calming and relaxing colours as the representation of agreeableness personality

\par We found a strong relationship between turquoise and \textbf{conscientiousness personality} dimension.
Generally, we can assume that turquoise color can be associated with nature which symbolizes green plants and clear sky.
Conscientiousness personality represents achievement-striving and concentration.
The calming effect of these colors could effect the decision of participant on measuring
conscientiousness higher than other four dimensions.

\par There is a positive relationship between blood-red and \textbf{neuroticism personality trait}.
Neuroticism personality is characterized by immoderation, depression, anxiety and etc which
makes them unable to approve themselves.
We think that since it is hard for neurotic to cope with stressful interpersonal situation,
it will try to avoid to be center of attention.
It can explain why such dark and dimmed shade of red as blood-red color conveys this personality trait.

\par Yellow screen color plays a significant role on portraying \textbf{openness personality trait}.
Openness dimension is characterized by deep imagination, openness to new knowledge,
having a wide spectrum of interests and curious.
Yellow color may be associated with the sun and being bright.
This color is quite visible and attention-grabbing color.
Thus, the openness mascot may trigger yellow color in order to be seen by others and
communicate with more new social devices.

\subsection{Discussion for the second study}
\label{subsec:discussion-for-the-second-study4}

Now considering each color separately, we can notice which personality traits they convey most.
In our second study, \textbf{yellow screen color} portrays openness personality
trait which makes the results from first study even more consistent.

\textbf{Orange screen color} shows the least effect on measuring personality trait of our mascot.
According to analysis, it is hard to discriminate one personality trait that would be conveyed by orange color.

\textbf{Turquoise screen color} mostly conveyed conscientiousness and agreeableness personality traits.

Whereas, \textbf{blood-red screen color} represents neuroticism personality trait which also confirms
the results from first study.

Generally, for \textbf{pink screen color} we have found a relationship with agreeableness personality trait.
However, in some sense, it also can be interpreted as conscientiousness personality.
Despite that, the pink color was mostly perceived as modest, forgiving, altruistic, unselfish mascot.
