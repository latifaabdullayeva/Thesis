\chapter{Discussions}
This chapter covers discussion of the results obtained from statistical tests. In our study, we tried to conceptualize the behavior of a mascot with four interaction types which we refer as use-cases. We discuss the results of each use in four sections separately. Moreover, as we mentioned in the Chapter 6, we analyzed results from two perspectives: the study of within personality trait and within condition. Each study splitted into subsections and discussed for each use-case separately.

\section{Mascot-Lamp interaction}
The study shows that there is a significant impact of the lamp emitting one of the predefined colors on the way how participants interpret the personality of a mascot. During the user study we explained participants that while interacting with the lamp, a mascot with its behavior tries to convey specific personality trait. Analysis shows that in this case participants did not just see the lamp as artificial light source, but also it gives an impression about mascot which triggered this light. In the following subsections we will discuss which lighting colors have a correlation to the particular personality traits.
\subsection{Discussion for the first study}
\par \textbf{Extraversion Personality Trait.} The results show that predefined lighting colors have an impact on the way how participants measured the personality of mascots. Almost 100\% of participants found a negative relationship between blood-red color and mascot's extraversion trait. Thus, by seeing this lighting color, participants did not reflect any positive opinion toward the Mascot measuring as friendly, warm, sociable, cheerful, positive and so on. Moreover, participants measured yellow color for a mascot which conveys positive, cheerful and all other characteristics that are close to extraversion personality traits. Unfortunately, since there is no difference found between yellow and orange colors after Bonferroni correction, we cannot conclude that yellow color is positively correlated to the extraversion. Despite that, with the help of the Bonferroni correction, the assumption that blood-red is measured as the opposite of all extraversion facets is consistent and does not include any false positives even for larger sample sizes. 
\par \textbf{Agreeableness Personality Trait.} The majority of participants had a positive opinion on attributing pink color to cooperative, trustworthy, modest mascot (see Figure-1). While the results for blood-red color showed the opposite direction implying the disagreement for this specific mascot to be Agreeable. Moreover, among all the colors, when the participants compared orange and pink and red and pink groups, their results were significantly different. They were giving the highest points to the mascot for being assessed as cooperative and modest in case of pink and the lowest points when they saw orange and blood-red lighting colors (see Figure-1). 
\par \textbf{Conscientiousness Personality Trait.} When measuring mascot for being conscientious, there was a clear separation of votes given for blood-red and orange with all other colors. Moreover, turquoise, pink and yellow, in descending order, were relatively similarly voted in a positive direction.
\par \textbf{Neuroticism Personality Trait.} Blood-red color achieves an excellent separation from all other colors having a sample of almost 100\% lying between 'neutral' and 'strongly accurate' scales showing a high perception of a mascot conveying being an angry, anxious, unstable behavior. While opinions for the pink color concentrated on the “inaccuracy” of this statement. Moreover, the mascots representing yellow, turquoise and especially pink lighting colors were measured as very low in the Neuroticism personality trait. Overall, the analysis revealed that by showing the participants blood-red together with any other colors, the majority will choose blood-red as an expected behavior for Neurotic Mascot. 
\par \textbf{Openness Personality Trait.} Overall, 75\% of the samples for yellow box-plot were concentrated between 'neutral' and 'accurate' toward mascot being interpreted as imaginative, adventurous, intelligent, having 25\% of votes condensed between 'accurate' and 'strongly accurate' scores. However, according to the statistical analysis (see Table-6.2), yellow color has a significant effect only when it is compared to blood-red and pink colors. Thus, during mascot lamp interaction, for Openness Mascot the yellow color is more preferable than the blood-red or pink. 

\subsection{Discussion for the second study}
\par \textbf{Yellow lighting color condition.} Since there is no positive relationship between yellow lighting and any of predefined personality traits. We can summarize that participants seeing yellow color will measure mascot as an extravert, or agreeable, or conscientious, or open to the experience. In addition, there is a low chance that the mascot representing the yellow lighting color will be perceived as neurotic. By knowing the significant negative correlation between variables, we can conclude with yellow lighting, the mascot will not be perceived as neurotic even for larger sample sizes.
\par \textbf{Orange lighting color condition.} When participants were shown orange lighting color, they measured mascots’ personality as both extravert or open to the experience. Thus, on the one hand, the mascot conveying the orange light was perceived as friendly, sociable, energetic, cheerful and the one which seeks excitement. On the other hand, this color is also attributed mascot that is imaginative, artistic, adventurous, intellectual. In addition, in comparison to yellow, turquoise and pink colors, orange color is not negatively correlated with neuroticism. This implies that to some extent if we broaden the survey, the orange lighting can convey the neuroticism personality trait of a mascot.
\par \textbf{Turquoise lighting color condition.} Overall, there is a positive relationship between conscientiousness and agreeableness personality traits and turquoise lighting. 
\par \textbf{Blood-red lighting color condition.} There is a positive relationship between blood-red and neuroticism personality trait. Thus, if we will choose a blood-red color, we will convey the neuroticism trait. In other words, the mascot triggering the blood-red lighting will be attributed as anxious, highly depressive, angry, vulnerable and the one that has immoderate behavior.
\par \textbf{Pink lighting color condition.} According to analysis, pink lighting color can convey the agreeableness or conscientiousness personality traits of a mascot. Thus, by presenting this lighting color, the mascot can be attributed as both modest, cooperative, trustworthy which are facets that constitute agreeableness, and orderly, dutiful, discipline which belongs to conscientiousness. Moreover, the negative relationship between pink and neuroticism implies that we can exclude that the mascot which triggers pink color will be attributed as anxious, vulnerable, angry and so on.

\section{Mascot-Speakers interaction}
For mascot-speakers interaction, we found a significant relationship between music and the personality trait on the mascot is presented by specific songs. We can report that there are some music categories that convey the particular personality trait of a mascot. In general, the interpretation of a mascot varies depending on which song this mascot triggered. In the following subsections we will discuss which music categories convey personality traits.
\subsection{Discussion for the first study}
\par \textbf{Extraversion Personality Trait.} Among all categories, Unpretentious has more consistent positive opinions for mascot being measured as an extravert device. However, there is no noticeable difference between Unpretentious and Contemporary music. Moreover, while assessing mascot’s extraversion personality trait, participants clearly distinguished Unpretentious from the Sophisticated category. Thus, participants tended to assess the mascot for which pop, rock\&roll, bluegrass songs were played, as a more friendly, sociable, cheerful personality. For these genres having noticeable rhythmic elements and a mainstream style, may lead participants to interpret the behavior of a mascot as a more energetic, gregarious and positive one.
\par \textbf{Agreeableness Personality Trait.} There is no statistically significant evidence that participants’ measurements differed between Sophisticated and Unpretentious music. However, for both of these categories participants attributed the higher scores of Agreeableness for a Mascot rather than for Contemporary. To conclude, there is a significant negative relationship between Contemporary music and the Agreeableness personality trait of a Mascot. The Sophisticated and Unpretentious music can be correlated positively with Agreeableness trait only when we compare each with Contemporary music genres.
\par \textbf{Conscientiousness Personality Trait.} Overall, the three music categories yield almost similar average scores, which explains the relatively less significant result of the Friedman test in comparison to other personality traits (see Table-6.3). Despite that, participants’ measurements regarding mascot conveying conscientiousness were significantly different when they were shown songs from Sophisticated and Contemporary categories. Thus, the genres from a Sophisticated category such as classical, jazz, and contemporary adult portrayed the mascot as being more disciplined, continuous, achievement-striving in comparison to rap and soul genres from Contemporary. Finally, there is a significant negative correlation between Contemporary music and the scores on facets that constitute a Conscientiousness personality trait.
\par \textbf{Neuroticism Personality Trait.} Among all three music categories, Contemporary showed a positive correlation with the Neuroticism personality trait of a mascot. In comparison to classical, jazz, pop, rock\&roll, and other genres, rap and soul describe mascot as being more emotionally unstable, angry, vulnerable, depressive. Participants explained this tendency as the following: mascots high in facets related to the Neuroticism personality captures aggression which is reflected in the music genres that they prefer to play. Participants who listened to Sophisticated and Unpretentious music measured mascot relatively low on Neuroticism.
\par \textbf{Openness Personality Trait.} Overall, the difference among all music categories is significantly different while assessing the openness personality trait of a mascot. This effect is substantially noticeable in the Sophisticated group which shows a positive correlation to a current personality trait. According to the participants’ opinions, classical, jazz songs contains ensembles of instruments with complex arrangements and they have distinctive tones and performance techniques which leads to measure the mascot as a complex, intelligent and thoughtful, the one who will enjoy non-mainstream music. 

\subsection{Discussion for the second study}
\par \textbf{Sophisticated music condition.} There is a positive correlation between sophisticated music and openness personality trait. In other words, the mascot that trigger speakers to play songs form jazz, classical or contemporary adult genres, will mostly convey openness personality trait. Moreover, the negative relationship between sophisticated and neuroticism implies that when participants listened to the above-mentioned genres, they measured the mascot very low in being anxious, aggressive, impulsive and all other facets represented by neuroticism.
\par \textbf{Contemporary music condition.} There is a connection between contemporary music and extraversion personality trait. Thus, in order to convey extravert mascot, such as rap and soul music can be helpful.
\par \textbf{Unpretentious music condition.} There is no positive relationship between any personality traits and Unpretentious music category. We observed only negative correlation between this category and neuroticism, meaning that songs as pop, rock\&roll, bluegrass and so on, can be interpreted very low on being aggressive, vulnerable, depressive and so on. The absence of differences between all other personality traits can imply that this music category may convey a mascot with a mix of being sociable, cheerful, trusting, cooperative, disciplined, orderly, artistic, intellectual and other facets belonging to extraversion, agreeable, conscientiousness and openness respectively. 

\section{Mascot-Mascot interaction}
The statistical analysis showed that the levels of vibration have a significant effect on the measurement of mascot’s personality. The vibration duration gives a clue about approaching mascot. Further, we discuss specific levels that personalities convey.

\subsection{Discussion for the first study}
\par \textbf{Extraversion Personality Trait.} The mascot with the longest vibration duration, namely 500 milliseconds long, was interpreted as assertive, forceful, energetic, friendly, sociable, cheerful. All of these facets constitute extraversion personality trait. Moreover, Level-4 showed a second-highest, after Level-1, positive correlation to the mascot being perceived as an extravert. In addition, after experiments, participants commented that the longer their mascot vibrated, the more they got an impression that approaching mascot wants to socialize with them. This made them give higher scores for extraversion facets. Importantly, participants were measuring all 30 facets without knowing which of them belong to the extraversion dimension. 
\par \textbf{Agreeableness Personality Trait.} Level-2 and Level-3 are more separated from all other levels showing a positive correlation for mascot being assessed as agreeable. This may indicate that for the majority of participants in order for mascot being assessed as modest, cooperative, trustworthy, the vibration level that they represent should not be as assertive as we have noticed for extraversion personality trait and not as passive as we will notice for neuroticism trait. 
\par \textbf{Conscientiousness Personality Trait.} Level-3 and Level-4 reveal a positive correlation to the currently described personality trait. Since conscientiousness personality trait is characterized by being goal-oriented (see Appendix Questionnaire), we can assume that participants might expect from conscientious mascot to be a little bit more assertive and forcefull in order to achieve its goals. Thus, it can explain why participants agreed that Level-4 conveys conscientious personality trait.
\par \textbf{Neuroticism Personality Trait.} The mascot was scored significantly high on being neurotic when participants experienced the vibration with the shortest duration, namely Level-1. This level sows a great separation from all other vibration levels. In addition, we can assume that the longer the duration of the vibration is, the more this mascot represents its desire to communicate with other mascots. Communication may require additional skills such as coping with a stressful situation, and in this kinds of situation, neurotic personality trait may be vulnerable and feel anxiety. It gleans insight into the reason why Level-1 is highly related to a neuroticism personality trait.
\par \textbf{Openness Personality Trait.} Overall, in comparison to other traits, the results for openness personality did not show very significant differences. The only difference were found between Level-1 and Level-2. Therefore, the mascot with a 200-milliseconds duration of the vibration may be perceived as an openness personality trait, only when it is compared with 100-milliseconds duration. The conclusion that we can draw is that such personality facets as imagination, adventurousness, intellect, liberalism hard to assess based on only vibration levels.

\subsection{Discussion for the second study}
\par \textbf{Level-1 vibration condition.} When the mascot approaches the participant that holds their own mascot, their mascots start to vibrate. When their mascot vibrated with the lowest level of vibration they interpreted approaching mascot as neurotic. Thus, Level-1 conveys anxious, depressive, vulnerable mascot.  
\par \textbf{Level-2 vibration condition.} Level 2 portrays modest, trusting, cooperative and straightforward facets which constitute agreeableness personality traits. This, if we want to assign agreeable personality traits to any social device the vibration level that is a good example to convey this personality is Level-2. We can assume that agreement could also tends to co-operate and socialize with others. Since is a mascot comprises modest facet, in order not to disturb other mascots or their owners, it communicate with others in the more restrained way. Thus, we expect the levels of vibration that has neutral duration such as Level-2 and later we will see Level-3. Because the higher is duration of vibration is, the more assertiveness it implies. Therefore, Level-2 does not disturb other owners of a mascot in an assertive manner which portrays the modesty and empathy traits of agreeableness.
\par \textbf{Level-3 vibration condition.} Level-3 may convey two personality traits: agreeableness and conscientiousness. the reason why we think Level-3 portrays agreeableness personality traits is the same as we discussed for Level-2. regarding conscientiousness personality, it is characterized by being orderly, dutiful, goal-oriented and therefore we would expect the tendency to be more careful with actions. This may explain that the neutral level of vibration conveys conscientiousness personality trait. In fact, some of these assumptions are made in order to better understand the reason behind participants' choices. Whereas, some of the assumptions are the comments that participants made in during video watch. Since our experiments did not include open-ended questionnaire, it can be a good idea to take it into account during future work. 
\par \textbf{Level-4 vibration condition.} According to results Level-4 also conveys two personality dimensions: conscientiousness and extraversion. The possible reason for Level-4 to portray conscientiousness personality trait might be the goal-oriented, achievement-striving and competence characteristics. On the one hand, Level-3 conveys conscientiousness personality trait because of its underliness, dutifulness, and self-discipline characteristics which we discussed above. On the other hand conscientiousness constitutes competence and achievements-striving facets which may be the reason why longer vibration duration conveys conscientiousness personality.
\par \textbf{Level-5 vibration condition.} Level 4 and level 5 they both portray extraversion personality dimension. The assertiveness and forceful facets of a mascot may be a reason why Level-5 conveys extraversion personality. We can assume that the mascot will force other mascot to vibrate with the longest duration until the owner react to it. The other reason might be the mascot is active, energetic and seeks for excitement, therefore, it tries to involve all other mascots and their owners to be social.

\section{Mascot-Tablet interaction}
Based on analysis, we can claim that some colors of the screen portrays specific personality traits. In the following subsections, we discuss which colors were more excessively assessed as personality traits. Moreover, we will give intuitive assumptions about why we think participants associated certain colors with particular personality traits. In fact, those assumptions are not bse on the research papers, they just help us to explain participants’ decisions in a more intuitive manner.

\subsection{Discussion for the first study}
\par There is no strong relationship between mascot being assessed high in \textbf{extraversion personality} and the background screen colour to the tablet displays. yellow and orange colours can be assessed as an extrovert mascot only when they are compared to pink and blood red colours. however it is hard to emphasise one colour that would convey extraversion personality.
\par The analysis revealed that blood-red screen color is negatively correlated with the mascot being measured as \textbf{agreeableness personality}. This implies that when participants have seen the tablet with blood red screen colour measured mascot as being very low in agreeableness. Moreover pink and turquoise colours convey agreeable personality. This can be explained by the agreeableness personality being kind, warm, generous, forgiving and unselfish and being portrayed by calming colours such as pink and turquoise. Pink colour can be associated with flowers and turquoise colour with sky or ocean for blue side of turquoise and nature for green side. Some participants mentioned that since these two colours remind nature, they decided to measure calming and relaxing colours as the representation of agreeableness personality
\par We found a strong relationship between turquoise and \textbf{conscientiousness personality} dimension. Generally, we can assume that turquoise color can be associated with nature which symbolizes green plants and clear sky. Conscientiousness personality represents achievement-striving and concentration. The calming effect of these colors could effect the decision of participant on measuring conscientiousness higher than other four dimensions.
\par There is a positive relationship between blood-red and \textbf{neuroticism personality trait}. Neuroticism personality is characterized by immoderation, depression, anxiety and etc which makes them unable to approve themselves. We think that since it is hard for neurotic to cope with stressful interpersonal situation, it will try to avoid to be center of attention. It can explain why such dark and dimmed shade of red as blood-red color conveys this personality trait.
\par Yellow screen color plays a significant role on portraying \textbf{openness personality trait}. Openness dimension is characterized by deep imagination, openness to new knowledge, having a wide spectrum of interests and curious. Yellow color may be associated with the sun and being bright. This color is quite visible and attention-grabbing color. Thus, the openness mascot may trigger yellow color in order to be seen by others and communicate with more new social devices.

\subsection{Discussion for the second study}
Now considering each color separately, we can notice which personality traits they convey most. In our second study, \textbf{yellow screen color} portrays openness personality trait which makes the results from first study even more consistent. \textbf{Orange screen color} shows the least effect on measuring personality trait of our mascot. According to analysis, it is hard to discriminate one personality trait that would be conveyed by orange color. \textbf{Turquoise screen color} mostly conveyed conscientiousness and agreeableness personality traits. Whereas, \textbf{blood-red screen color} represents neuroticism personality trait which also confirms the results from first study. Generally, for \textbf{pink screen color} we have found a relationship with agreeableness personality trait. However, in some sense, it also can be interpreted as conscientiousness personality. Despite that, the pink color was mostly perceived as modest, forgiving, altruistic, unselfish mascot.
