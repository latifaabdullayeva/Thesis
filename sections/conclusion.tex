\chapter{Conclusion}
\label{ch:conclusion}
Section 8.1 of the final chapter gives a conclusive overview of the results achieved during conducting this research.
In section 8.2, we describe the contribution and our attempts to improve the system compared to related
work [Chapter 2] and in section 8.3 we provide certain limitations to our study.
Further, in section 8.4, we discuss how the ideas explored in this project can be continued further.

\section{Overview}
\label{sec:overview}
In this study both psychological and user‐experience approaches helps us to better understand thing to thing interaction.

\subsection{The birth of the idea}
\label{subsec:the-birth-of-the-idea}
In our study we started with a question: “What interactive objects with a certain behavior can tell us about itself?”.
Thus, we decided to assign a personality trait to a mascot, assuming that personality
may be identification of this social device.
We hope that the interaction between a mascot and other inanimate devices might be a
descriptive clue for personality of that mascot.
Therefore, the actions that mascot is taking might convey the personality of it.
From analysis point of view, taking into account all possible types of interaction
would cause a huge data set that will be hard to analyze.
Thus, we decided to minimize
possible actions by using Proxemics theory, which helped us to come up with case-studies such
as the interaction between mascot-mascot, mascot-tablet, mascot-lamp, and mascot-speakers.
Further, for each interaction type we had to come up with possible actions such as mascot
triggering specific song or lighting color and so on.
During our study, we have noticed that
certain type of behavior, for example, mascot emitted pink color or
sophisticated song and so on, conveys a particular personality trait.

\subsection{Overview of the results}
\label{subsec:overview-of-the-results}
Now we can review the final results.
We analyzed the data from two perspectives and reported them as study-1 and study-2 in chapter 6.
In the first study, we observe the personality trait and the variation of all
conditions within each personality individually.
In our second study, we take a closer look at each condition and the variation of
all personality traits within a specific condition.
The second study is supplementary for the first study.
When during the first study, we see that one condition conveys several personality traits,
the second study gives us additional information regarding which of these personality
traits conveyed most by this condition.

\par For \textbf{Mascot-lamp interaction}, we can conclude that both studies showed that
there is a positive relationship between blood-red and neuroticism personality.
Thus, if the mascot triggers blood-red lighting as a representation of its behavior,
it will convey the neuroticism personality trait.
Choosing turquoise or pink lighting as a behavior of the
mascot will be measured either as a conscientious or agreeable social device.
The same tendency is observed for orange color, by showing this color, we may convey
either extraversion or openness personality trait.
Unfortunately, for mascot-lamp interaction, yellow lighting failed to portray stable personality trait.

\par For \textbf{Mascot-speakers interaction}, in order to convey extraversion or neuroticism personality traits,
contemporary music will be the best choice.
For the mascot that attributes an openness personality trait,
the good choice will fall on sophisticated music.
Or the other way around, by choosing sophisticated music, the personality trait that
will be conveyed is an openness.
When we choose contemporary music, the mascot will convey an extraversion personality trait.
In our first study, we notice that both extraversion and neuroticism are positively related to contemporary music.
Having two personalities related to the same music category, our second study
revealed that when choosing contemporary music, extraversion personality will be conveyed more.

\par For \textbf{Mascot-mascot interaction}, extraversion mascot can be presented by
showing higher levels of vibration, namely Level-4 and Level-5.
If we assign agreeableness personality to our mascot or any other social device,
the good choice will be focusing on Level-2 and Level-3.
For conscientious mascot, we can take Level-3 and Level-4 as a representational behavior of that personality trait.
When we want to convey the neuroticism personality trait, we make a choice for Level-1,
the vibration with the lowest duration.
And vice versa, as a result of the second study, Level-1 will convey that the mascot
is neurotic, Level-2 - agreeable, and Level-5 that the mascot is an extravert.
Moreover, by choosing Level-3, we will able to portray either agreeableness or conscientious personality traits.
And by choosing Level-4, both extraversion and conscientiousness personalities can be conveyed.

\par For \textbf{Mascot-tablet interaction}, in order to convey the conscientiousness personality trait,
mascot should be able to trigger a tablet to display turquoise color.
Whereas, for a neurotic mascot, the blood-red may be a good and distinctive choice.
Finally, when we want to convey an openness personality trait, the mascot should represent yellow color.
Further, if we want to focus on colors, yellow screen color is a good choice to convey openness,
blood-red portrays neuroticism personality trait.
In addition, turquoise and pink screen colors can exhibit either agreeableness or
conscientious personality traits.
The last two colors may need further study.

\par Despite the fact that yellow lighting could not convey any personality traits for
mascot-lamp interaction, there is a clear positive relationship between yellow screen color
and mascot’s openness personality trait for mascot-tablet interaction.
We can assume that the reason for a different interpretation of the same color in different
case-studies is that the lamp emitting yellow light can be perceived as the
typical color you get from incandescent bulbs.
Meanwhile, the yellow screen color is perceived as a more vivid color framed in a screen.
The same observation is made for blood-red color which conveys a neurotic personality trait for both case-studies.
However, the mascot triggering the lamp with blood-red lighting is perceived as
more angry, anxious, depressive, emotionally unstable and impulsive than the
mascot triggering the same color in the tablet.
We can assume that by illuminating the whole room with blood-red color people can
get a more negative impression about mascot’s personality than seeing the same color in a tablet-size screen.

\section{Contributions and findings}
\label{sec:contributions-and-findings}
The main contribution is an achieving cooperation among mascot and other interactive
objects in a system where each mascot has a unique personality trait.
This can, in turn, serve as a contribution for designing social devices in SIoT environment.
Moreover, the networthing finding is that there is a relationship between
personality trait and behavior of interactive objects.
Although this relationship is not as clear and vivid as we observe in human-human interaction,
it is a good insight that personality and behavior are interconnected concepts even for inanimate objects.
These concepts can help engineers to design their social devices and to serve as an
aid for users to better understand interaction between them.
For example, when researchers or engineers would like to construct SIoT devices
with certain personalities they could use our findings about types of behavior that can
convey this personality trait.
We hope that behavior that we came up with, particularly four types of interaction shed
a light on how behavior of social devices can help people to understand their personality and
therefore, the main goals, intentions, and motivation of inanimate object.
When engineers try to configure personality on social device in the context of behaviour and
interaction types that we described they can expect actions that we reported in chapter 6.
On the other hand if researchers or engineers set up specific commands i.e actions to the
social device the users will understand the device's personality as we described in discussion chapter.

\par Moreover, there is one unexpected phenomenon that we observed during experiments.
We know that each participant has its own approach on measuring the personality traits of a mascot.
So, before experiment, we believed that while measuring the mascot’s personality,
participants will rely on their background knowledge of non-verbal clues such as favorite colors or music and so on.
However, during the experiments, we noticed that some participants did not focus on
factors such as lighting, music and etc as pure independent variable that is affecting their choice.
During post-experiment discussions, we noticed that they first associated these factors with
someone from their life and then analyze the mascot as they would assess that person.
This was an interesting finding showing that some people would prefer to measure the
personality of social device by relating it to the real world scenarios and making
references to the person from real-world.
Of course, this observation is hard to extend to a whole population.
However, it was an easter egg in a sense that when people were observing the
interaction between social devices, they immediately applied anthropomorphic rules
even if they knew that these are inanimate objects.

\section{Limitations}
\label{sec:limitations}
Unfortunately, the study was based on relatively small (N=25) and homogeneous
(e.g. having 70\% of participants from formal sciences and age mean of 26) samples which
limited the power of their analyses and make it difficult to extrapolate findings to a general population.
However, the experiment design was counterbalanced in terms of other important
characteristics such as gender, participants’ overall music preferences for mascot-speakers
interaction and the order in which participants watched the videos.

\par Another limitation could be that the study was held in a laboratory setting where
participants only had a limited time to assess the personalities of mascots.
On the one hand, there is a possibility that spending more time (i.e. hours or days)
with mascots could affect participant’s opinions on assigning the personality trait to these devices.
On the other hand, during these experiments participants reflected their very first impressions
while seeing the interaction between devices and their initial reaction while measuring the
personality traits of mascot based on these interactions.

\par The results did not reveal full pattern where we can state that certain personality
can be conveyed with the following color, vibration level and music genre.
However, we shed a light on the impact of these factors can have an perceiving the personality of social device.
Despite the fact that some of the factors were hard to measure as a behavior of certain personalities,
Table(conclusion) gave us an insight that personality of social devices and their behavior
are interconnected concepts.
Moreover, some of these personality traits can be ??vividly conveyed by pre-defined factors.


\section{Future work}
\label{sec:future-work}
\par In future work, design researchers interested in contributing to our study can focus
on finer specifications of factors.
These specifications can be either toward new interaction types and behavior or toward
improvement of the results from existing interactions.
Testing more songs from different genres, or more colors for light and screen may serve as a good example.
Another idea might be improving mascot-tablet interaction in addition to screen colors
some other functionalities can be added.
For example, when mascot approaching tablet it can show some favorite pictures in order
to give more information about mascot.
In fact, the design of such system should also take into account security aspects for not leaking personal photos.

\par Another approach might be using the interaction between social devices as a trigger or
motivation for people to socialize and communicate more.
The behavior of mascots such as trigger lighting or music play or vibrate some other
person’s mascot when you approach it, can be an icebreaker for people to communicate with each other.